%%% Design basiert auf:
%% LaTeX-Beamer template for KIT design
%% by Erik Burger, Christian Hammer
%% title picture by Klaus Krogmann
%% version 2.1
\documentclass[18pt]{beamer}

\usepackage{templates/beamerthemekit}
\usepackage{graphicx} %because it will be included below
\usepackage{listings}
%\usepackage{wasysym}
\usepackage{color}
\usepackage[T1]{fontenc}
%\usepackage{cmap}
\usepackage{textcomp}
\usepackage[utf8]{inputenc}
%\usepackage{pdfpages}
\definecolor{listinggray}{gray}{0.9}
\definecolor{lbcolor}{rgb}{0.9,0.9,0.9}
\lstset{
        language=Java,
        backgroundcolor=\color{lbcolor},
        tabsize=4,
		keepspaces,
		extendedchars=true,
        rulecolor=\color{black},
        basicstyle=\footnotesize,
        aboveskip=0pt,
        upquote=true,
        columns=fixed,
        showstringspaces=false,
        extendedchars=true,
        breaklines=true,
        frame=single,
        showtabs=false,
        showspaces=false,
        showstringspaces=false,
        identifierstyle=\ttfamily,
        keywordstyle=\color[rgb]{0,0,1},
        commentstyle=\color[rgb]{0.133,0.545,0.133},
        stringstyle=\color[rgb]{0.627,0.126,0.941},
}


%% TITLE PICTURE
\titleimage{frontpic}


% For the title page
\title[Proggen WS11/12]{Programmieren WS 2011/2012}
\subtitle{Tutorium Nr. 1 / 11}
\author{Tobias Sturm} %, David Kulicke
\institute{Zertifizierbare Vertrauenswürdige Informatiksysteme}
\date[23.1.12] %TODO aktualisieren

% the presentation starts here
\begin{document}
\selectlanguage{ngerman}


%title page
\begin{frame}
	\titlepage
\end{frame}


%table of contents
\begin{frame}{Heute:}
%	\setcounter{tocdepth}{1}
	\tableofcontents
\end{frame}

\setbeamercovered{invisible}

%%%%%%%%%%%%%%%%%%%%%%%%%%%%%%%%%%%%%%%%%%%%%%%%%%%%%%%%%%%%%%%%%%%%%%%%%
%%%%%%%%%%%%%%%%%%%%%%%%%%%%%%%%%%%%%%%%%%%%%%%%%%%%%%%%%%%%%%%%%%%%%%%%%
%%%%%%%%%%%%%%%%%%%%%%%%%%%%%%%%%%%%%%%%%%%%%%%%%%%%%%%%%%%%%%%%%%%%%%%%%
\section{Arrays}
\subsection{Was ist das?}
\begin{frame}{Was sind Arrays ...}
	\textbf{... und wozu braucht man sie?} \pause
		\begin{itemize}
			\item viele Werte in einem Variablennamen
			\item Elemente haben alle den selben Wert
		\end{itemize}
		
		$\rightarrow$ Zu jeden \emph{Typen} gibt es ein Array.\pause
		
		Wichtig: Arrays sind komplexe Datentypen! $\rightarrow$ new verwenden!
\end{frame}

%%%%%%%%%%%%%%%%%%%%%%%%%%%%%%%%%%%%%%%%%%%%%%%%%%%%%%%%%%%%%%%%%%%%%%%%%

\begin{frame}[containsverbatim]
	\frametitle{Arrays ...}
		
		... deklarieren
		\begin{lstlisting}
int[] myArray; //Integer-Array
		\end{lstlisting}

		... deklarieren und instanzieren
		\begin{lstlisting}
int[] myArray = new int[3]; //Arrays with three int values
		\end{lstlisting}
		
		... deklarieren und initialisieren
		\begin{lstlisting}
int[] myArray = {5, 3, 1}; //Define the values of the array -> size implicit
		\end{lstlisting}

		\pause !!! Kommentare auf dieser Folie sind \textbf{NICHT} gut !!!
\end{frame}

%%%%%%%%%%%%%%%%%%%%%%%%%%%%%%%%%%%%%%%%%%%%%%%%%%%%%%%%%%%%%%%%%%%%%%%%%

\begin{frame}[containsverbatim]
	\frametitle{verschiedene Arrays}
		
		\begin{lstlisting}
double[] myArr1; //array of doubles
Teddy[] myTeddyArr; //array of Teddys
String[] myStrArray; //array of Strings
		\end{lstlisting}
		
\end{frame}

%%%%%%%%%%%%%%%%%%%%%%%%%%%%%%%%%%%%%%%%%%%%%%%%%%%%%%%%%%%%%%%%%%%%%%%%%

\begin{frame}[containsverbatim]
	\frametitle{mit Arrays arbeiten}
		
		\begin{lstlisting}
String[] names = new String[4];
names[0] = "Timmy"; //defines value of the FIRST element
names[3] = "Bernd"; //Sets the LAST element
names[4] = "Alice"; //will throw a IndexOutOfBoundException
		\end{lstlisting}
		
\end{frame}

%%%%%%%%%%%%%%%%%%%%%%%%%%%%%%%%%%%%%%%%%%%%%%%%%%%%%%%%%%%%%%%%%%%%%%%%%


\begin{frame}[containsverbatim]
	\frametitle{Einschub Exceptions}
		
		Bei Java gibt es
		\begin{itemize}
			\item Compiler-Fehler (Syntax)
			\item Laufzeit-Fehler (Exceptions)
		\end{itemize}
		
		\pause	Laufzeitfehler sind abhängig vom Verlauf des Programs und können meistens nicht vorhergesehen werden.
\end{frame}

%%%%%%%%%%%%%%%%%%%%%%%%%%%%%%%%%%%%%%%%%%%%%%%%%%%%%%%%%%%%%%%%%%%%%%%%%

\subsection{Arrays sind Objekte}
\begin{frame}
	\frametitle{Arrays sind Objekte}
		
		$\rightarrow$ deshalb haben sie Methoden und Attribute.\pause
		
		
		z.B. \emph{length} $\rightarrow$ gibt die Anzahl der Elemente zurück
		
\end{frame}


%%%%%%%%%%%%%%%%%%%%%%%%%%%%%%%%%%%%%%%%%%%%%%%%%%%%%%%%%%%%%%%%%%%%%%%%%

\subsection{Mit Arrays arbeiten}
\begin{frame}[containsverbatim]
	\frametitle{Aufgabe}
		Wir haben ein Array mit Zahlen gegeben und wollen den Durchschnitt errechnen.
\end{frame}

%%%%%%%%%%%%%%%%%%%%%%%%%%%%%%%%%%%%%%%%%%%%%%%%%%%%%%%%%%%%%%%%%%%%%%%%%

\begin{frame}[containsverbatim]
	\frametitle{Aufgabe}
		Wir haben ein Array mit Flughafen-Kürzeln gegeben und möchten den Inhalt auf der Konsole ausgeben.
\end{frame}

%%%%%%%%%%%%%%%%%%%%%%%%%%%%%%%%%%%%%%%%%%%%%%%%%%%%%%%%%%%%%%%%%%%%%%%%%

\begin{frame}[containsverbatim]
	\frametitle{Aufgabe}
		Jetzt wollen wir das Flughafen-Kürzel-Array in umgekehrter Reihnfolge ausgeben.
\end{frame}

%%%%%%%%%%%%%%%%%%%%%%%%%%%%%%%%%%%%%%%%%%%%%%%%%%%%%%%%%%%%%%%%%%%%%%%%%

\begin{frame}[containsverbatim]
	\frametitle{Aufgabe}
		Jetzt wollen wir jedes dritte Element ausgeben.
\end{frame}

%%%%%%%%%%%%%%%%%%%%%%%%%%%%%%%%%%%%%%%%%%%%%%%%%%%%%%%%%%%%%%%%%%%%%%%%%

\begin{frame}[containsverbatim]
	\frametitle{Arrays kopieren}
		
		\textbf{was passiert hier?}
		\begin{lstlisting}
String[] myArr1 = {"geh", "du", "alter", "esel"};
String[] myArr2 = myArr1;
myArr2[3] = "sack";
for (int i = 0; i < myArr1.length; i++) {
	System.out.println(myArr1[i] + " ");
}
		\end{lstlisting}
		
\end{frame}

%%%%%%%%%%%%%%%%%%%%%%%%%%%%%%%%%%%%%%%%%%%%%%%%%%%%%%%%%%%%%%%%%%%%%%%%%

\begin{frame}[containsverbatim]
	\frametitle{Arrays kopieren}
		\textbf{So funktioniert's:}
		\begin{lstlisting}
String[] myArr1 = {"geh", "du", "alter", "esel"};
String[] myArr2 = new String[myArr1.length];

//makes a deep copy myArr1 -> myArr2
for (int i = 0; i < myArr1.length; i++) {
	myArr2[i] = myArr[i];
}
		\end{lstlisting}
		
\end{frame}

%%%%%%%%%%%%%%%%%%%%%%%%%%%%%%%%%%%%%%%%%%%%%%%%%%%%%%%%%%%%%%%%%%%%%%%%%

\begin{frame}[containsverbatim]
	\frametitle{Aufgabe}
		Zurück zum Flughafen-Array:
		Jetzt wollen wir das Array in umgekehrter Reihnfolge kopieren.
\end{frame}

%%%%%%%%%%%%%%%%%%%%%%%%%%%%%%%%%%%%%%%%%%%%%%%%%%%%%%%%%%%%%%%%%%%%%%%%%

\begin{frame}[containsverbatim]
	\frametitle{Mehrdimensionale Arrays}
		
		\begin{lstlisting}
boolean[][] map = new boolean[10][10];
		\end{lstlisting}
		
		\textbf{Was ergeben diese Ausdrücke:}
		\begin{itemize}
			\item map.length
			\item map[0]
		\end{itemize}
		
\end{frame}

%%%%%%%%%%%%%%%%%%%%%%%%%%%%%%%%%%%%%%%%%%%%%%%%%%%%%%%%%%%%%%%%%%%%%%%%%

\begin{frame}[containsverbatim]
	\frametitle{Mehrdimensionale Arrays}
		
		\begin{lstlisting}
boolean[][] map = new boolean[10][10];
		\end{lstlisting}
		
		\textbf{Aufgabe:} Wir wollen alle Werte auf der Diagonalen auf true setzen, alle anderen auf false
\end{frame}


%%%%%%%%%%%%%%%%%%%%%%%%%%%%%%%%%%%%%%%%%%%%%%%%%%%%%%%%%%%%%%%%%%%%%%%%%

\begin{frame}[containsverbatim]
	\frametitle{Inception-Arrays}
		
		Wer hat am Abend wie viel getrunken?
		\begin{lstlisting}
double[] drinkListAnna = {0.3, 0.25, 0.5, 0.02};
double[] drinkListClaus = { };
double[] drinkListPeter = {0.5, 0.5, 0.5, 0.02, 0.5, 0.5, 0.5, 0.02, 0.02};

double[][] party = {drinkListAnna, drinkListClaus, drinkListPeter};
		\end{lstlisting}
		
		\textbf{Aufgabe:} Liste aller Getränke auflisten
\end{frame}

%%%%%%%%%%%%%%%%%%%%%%%%%%%%%%%%%%%%%%%%%%%%%%%%%%%%%%%%%%%%%%%%%%%%%%%%%

\begin{frame}[containsverbatim]
	\frametitle{Arrays vergrößern}
		
		\begin{lstlisting}
String[] guestList = new String[4];
guestList[0] = "Peter";
guestList[1] = "Susi";
guestList[2] = "Franz";
guestList[3] = "Chuck Norris";
		\end{lstlisting}
		
		\textbf{Aufgabe:} Schreibe eine Methode, mit der man die Gästeliste um eine Person erweitern kann.
\end{frame}

%%%%%%%%%%%%%%%%%%%%%%%%%%%%%%%%%%%%%%%%%%%%%%%%%%%%%%%%%%%%%%%%%%%%%%%%%
%%%%%%%%%%%%%%%%%%%%%%%%%%%%%%%%%%%%%%%%%%%%%%%%%%%%%%%%%%%%%%%%%%%%%%%%%

\section{Aufgabenblatt}
\subsection{Aufgabenblatt}
\begin{frame}{Aufgabenblatt}
	
\end{frame}

%%%%%%%%%%%%%%%%%%%%%%%%%%%%%%%%%%%%%%%%%%%%%%%%%%%%%%%%%%%%%%%%%%%%%%%%%
%%%%%%%%%%%%%%%%%%%%%%%%%%%%%%%%%%%%%%%%%%%%%%%%%%%%%%%%%%%%%%%%%%%%%%%%%
%%%%%%%%%%%%%%%%%%%%%%%%%%%%%%%%%%%%%%%%%%%%%%%%%%%%%%%%%%%%%%%%%%%%%%%%%

\section{Zusammenfassung}
\subsection{Zusammenfassung}
\begin{frame}{Zusammenfassung}

	Arrays ...
	\begin{itemize}
		\item ... machen
		\item ... füllen
		\item ... durchlaufen
		\item ... kopieren
	\end{itemize}
\end{frame}%%%%%%%%%%%%%%%%%%%%%%%%%%%%%%%%%%%%%%%%%%%%%%%%%%%%%%%%%%%%%%%%%%%%%%%%%%%%%%%%%%%%%%%%%%%%%%%%%%%%%%%%%%%%%%%%%%%%%%%%%%%%%%%%%%%%%%%%%%%%%%%%%%

%Noch fragen Folie
\section{Fragen?}
\subsection*{Fragen} %Für das Design...
\begin{frame}	
	\begin{center}
		\huge{Fragen?}
	\end{center}
\end{frame}



%comic
\begin{frame}[full]
\includegraphics[scale=0.55]{bilder/comics/September-25-2011-18-44-59-aa71ce1bd67502c27bc56a6b8d724897.jpeg}
\end{frame}
\end{document}

\end{document}