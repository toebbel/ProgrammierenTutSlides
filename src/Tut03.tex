%%% Design basiert auf:
%% LaTeX-Beamer template for KIT design
%% by Erik Burger, Christian Hammer
%% title picture by Klaus Krogmann
%% version 2.1
\documentclass[18pt]{beamer}

\usepackage{templates/beamerthemekit}
\usepackage{graphicx} %because it will be included below
\usepackage{listings}
%\usepackage{wasysym}
\usepackage{color}
\usepackage[T1]{fontenc}
%\usepackage{cmap}
\usepackage{textcomp}
\usepackage[utf8]{inputenc}
%\usepackage{pdfpages}
\definecolor{listinggray}{gray}{0.9}
\definecolor{lbcolor}{rgb}{0.9,0.9,0.9}
\lstset{
        language=Java,
        backgroundcolor=\color{lbcolor},
        tabsize=4,
		keepspaces,
		extendedchars=true,
        rulecolor=\color{black},
        basicstyle=\footnotesize,
        aboveskip=0pt,
        upquote=true,
        columns=fixed,
        showstringspaces=false,
        extendedchars=true,
        breaklines=true,
        frame=single,
        showtabs=false,
        showspaces=false,
        showstringspaces=false,
        identifierstyle=\ttfamily,
        keywordstyle=\color[rgb]{0,0,1},
        commentstyle=\color[rgb]{0.133,0.545,0.133},
        stringstyle=\color[rgb]{0.627,0.126,0.941},
}


%% TITLE PICTURE
\titleimage{frontpic}


% For the title page
\title[Proggen WS11/12]{Programmieren WS 2011/2012}
\subtitle{Tutorium Nr. 1 / 11}
\author{Tobias Sturm} %, David Kulicke
\institute{Zertifizierbare Vertrauenswürdige Informatiksysteme}
\date[23.1.12] %TODO aktualisieren

% the presentation starts here
\begin{document}
\selectlanguage{ngerman}


%title page
\begin{frame}
	\titlepage
\end{frame}


%table of contents
\begin{frame}{Heute:}
%	\setcounter{tocdepth}{1}
	\tableofcontents
\end{frame}

\setbeamercovered{invisible}

%%%%%%%%%%%%%%%%%%%%%%%%%%%%%%%%%%%%%%%%%%%%%%%%%%%%%%%%%%%%%%%%%%%%%%%%%
%%%%%%%%%%%%%%%%%%%%%%%%%%%%%%%%%%%%%%%%%%%%%%%%%%%%%%%%%%%%%%%%%%%%%%%%%
%%%%%%%%%%%%%%%%%%%%%%%%%%%%%%%%%%%%%%%%%%%%%%%%%%%%%%%%%%%%%%%%%%%%%%%%%
\section{Ausdrücke und Anweisungen}
\subsection{Ausdrücke}
\begin{frame}
	\frametitle{Ausdrücke ...}
	\framesubtitle{richtig harte Ausdrücke ...}
	
	\textbf{Was war das nochmal?} \pause
	
		\begin{itemize}
			\item ... ergeben einen Wert/Aussage
			\item ... haben einen Datentyp
			\item ... können geschachtelt werden
			\item ... können aneinandergereit werden
		\end{itemize}
\end{frame}

%%%%%%%%%%%%%%%%%%%%%%%%%%%%%%%%%%%%%%%%%%%%%%%%%%%%%%%%%%%%%%%%%%%%%%%%%

\begin{frame}[containsverbatim]
	\frametitle{Ausdrücke}
	\framesubtitle{richtig harte Ausdrücke ...}
	
		\textbf{Aufgabe:} Finde die Ausdrücke	
		\begin{lstlisting}
int x = 5;
y = x;
y++;
int z = addiere(x, y) + 1;
double d = addiere(addiere(x, y), addiere(4, z));
if(d == 27)
	System.out.println("wow!");
		\end{lstlisting}

\end{frame}

%%%%%%%%%%%%%%%%%%%%%%%%%%%%%%%%%%%%%%%%%%%%%%%%%%%%%%%%%%%%%%%%%%%%%%%%%
%%%%%%%%%%%%%%%%%%%%%%%%%%%%%%%%%%%%%%%%%%%%%%%%%%%%%%%%%%%%%%%%%%%%%%%%%

\subsection{Anweisungen}
\begin{frame}
	\frametitle{Anweisungen ...}

		\pause
		\begin{itemize}
			\item ... haben keine Rückgabe / Typ
			\item ... führen Zustandsänderungen durch
		\end{itemize}
\end{frame}

\begin{frame}[containsverbatim]
	\frametitle{Anweisungen}
	\framesubtitle{richtig harte Anweisungen ...}
	
		\textbf{Aufgabe:} Finde die Anweisungen
		\begin{lstlisting}
int x = 5;
y = x;
int z = addiere(x, y);
double d = addiere(addiere(x, y), addiere(4, z));\end{lstlisting}
		
		
		
		\pause $\rightarrow$ Alle Ausdrücke sind Teil einer Anweisung!
\end{frame}

%%%%%%%%%%%%%%%%%%%%%%%%%%%%%%%%%%%%%%%%%%%%%%%%%%%%%%%%%%%%%%%%%%%%%%%%%
%%%%%%%%%%%%%%%%%%%%%%%%%%%%%%%%%%%%%%%%%%%%%%%%%%%%%%%%%%%%%%%%%%%%%%%%%
%%%%%%%%%%%%%%%%%%%%%%%%%%%%%%%%%%%%%%%%%%%%%%%%%%%%%%%%%%%%%%%%%%%%%%%%%

\section{Abstraktion}
\subsection{Anonyme Objekte}
\begin{frame}[containsverbatim]
	\frametitle{Anonyme Objekte}
		\emph{Kreis.java}
		\begin{lstlisting}
public class Kreis {
	int x, y, radius;
	public Kreis(int x, int y, int radius) {
		this.x = x;
		this.y = y;
		this.radius = radius:
	}
}
		\end{lstlisting}
		
		\emph{Main.java}
		\begin{lstlisting}
//...
Kreis myKreis = new Kreis(10, 15, 3);
//...
		\end{lstlisting}
		
		Was könnte man hier besser machen
\end{frame}

%%%%%%%%%%%%%%%%%%%%%%%%%%%%%%%%%%%%%%%%%%%%%%%%%%%%%%%%%%%%%%%%%%%%%%%%%

\begin{frame}[containsverbatim]
	\frametitle{Anonyme Objekte}
	
	\emph{Kreis.java}
	\begin{lstlisting}
public class Kreis {
	int radius;
	Punkt mittelpunkt;
	public Kreis(Punkt mp, int radius) {
		this.radius = radius;
		mittelpunkt = mp;
	}
}
		\end{lstlisting}
	
		\emph{Main.java}
		\begin{lstlisting}
//...
Punkt myPunkt = new Punkt(10, 15);
Kreis myKreis = new Kreis(myPunkt, 3);
//...
		\end{lstlisting}
\end{frame}

%%%%%%%%%%%%%%%%%%%%%%%%%%%%%%%%%%%%%%%%%%%%%%%%%%%%%%%%%%%%%%%%%%%%%%%%%

\begin{frame}[containsverbatim]
	\frametitle{Anonyme Objekte}
	\emph{Kreis.java}
		
	... bleibt unverändert	
	
	
	
		\emph{Main.java}
		\begin{lstlisting}
//...
Kreis myKreis = new Kreis(new Punkt(10, 15), 3);
//...
		\end{lstlisting}
		
		Wir brauchen die Variable \emph{myPunkt} nicht mehr, nachdem wir unseren Kreis erzeugt haben $\rightarrow$ Warum sollten wir ihr einen Namen geben?
\end{frame}

%%%%%%%%%%%%%%%%%%%%%%%%%%%%%%%%%%%%%%%%%%%%%%%%%%%%%%%%%%%%%%%%%%%%%%%%%
%%%%%%%%%%%%%%%%%%%%%%%%%%%%%%%%%%%%%%%%%%%%%%%%%%%%%%%%%%%%%%%%%%%%%%%%%

\subsection{Getter-Methoden}
\begin{frame}[containsverbatim]{Geheimnisprinzip}
\begin{lstlisting}
class Student {
	String name;
	int semester;
	int matriculationNumber;
	
	//Konstruktor usw.
}
\end{lstlisting}
\begin{lstlisting}
class Main {
	public static void main(String[] args) {
		//einen neuen Studenten erzeugen
		Student maxMustermann = new Student("Max Mustermann", 3, 1234567);
		//die Matrikelnummer laesst sich ganz einfach aendern
		maxMustermann.matriculationNumber = 4545454;
		//...
\end{lstlisting}
\ \\ \ \\
\end{frame}

\begin{frame}[containsverbatim]{Geheimnisprinzip}
\begin{lstlisting}
class Student {
	String name;
	int semester;
	int matriculationNumber;
	
	//Konstruktor usw.
}
\end{lstlisting}
\begin{lstlisting}
class Main {
	public static void main(String[] args) {
		//einen neuen Studenten erzeugen
		Student maxMustermann = new Student("Max Mustermann", 3, 1234567);
		//die Matrikelnummer laesst sich ganz einfach aendern
		maxMustermann.matriculationNumber = 4545454;
		//...
\end{lstlisting}
Die Matrikelnummer sollte doch nachträglich \textbf{nicht} verändert werden können...
\end{frame}

\begin{frame}{Das Problem:}
\begin{alertblock}{}
Bis jetzt können wir ungehindert auf alle Attribute eines Objekts zugreifen und sie verändern, was nicht immer so toll ist.
\end{alertblock}

\pause
\begin{block}{Mögliche Lösungen:}
\begin{itemize}
	\item Es einfach hinnehmen \pause (schlecht)
	\pause
	\item den Benutzer freundlich mit einem Kommentar darauf hinweisen, bestimmte Attribute nicht zu ändern
	\pause (auch nicht wirklich gut...)
	\pause
	\item \textbf{Zugriffsmodifikatoren} einsetzen!
\end{itemize}
\end{block}
\end{frame}

\begin{frame}{Sichtbarkeiten in Java}
\begin{block}{Zugriffsmodifikatoren}
Mit Hilfe von \textbf{Zugriffsmodifikatoren} (access modifiers) lassen sich die \textbf{Sichtbarkeiten} von Programmteilen regeln:
\begin{itemize}
	\item \textbf{public} \textit{Element}: Element ist für alle Klassen sichtbar
	\item \textbf{private} \textit{Element}: Element ist nur innerhalb seiner Klasse sichtbar
	\pause
	\item  \textbf{protected} \textit{Element}: Element ist nur innerhalb seiner Klasse, deren Subklassen und allen Klassen im selben Paket sichtbar\\
	$\rightarrow$später mehr dazu
	\item \textbf{kein Modifier}: Element ist nur innerhalb seiner Klasse und der Klassen im selben Paket sichtbar\\
	$\rightarrow$hier nicht so wichtig
\end{itemize}
\end{block}
\end{frame}

\begin{frame}[containsverbatim]{Sichtbarkeiten in Java}
\begin{lstlisting}
public class Student {
	//die Attribute sind nun nach aussen nicht mehr sichtbar
	private String name;
	private int semester;
	private int matriculationNumber;

	public Student(String name, int semester, int matriculationNumber) {
	//hier wird wie gewohnt alles initialisiert
	}	
}
\end{lstlisting}
\begin{lstlisting}
public class Main {
	public static void main(String[] args) {
		Student maxMustermann = new Student("Max Mustermann", 3, 1234567);
		//hier bekommt man nun einen Compilerfehler
		maxMustermann.matriculationNumber = 4545454;
		//...
\end{lstlisting}
\end{frame}

\begin{frame}{Sichtbarkeiten in Java}
\setbeamercovered{invisible}
\begin{center}{\huge Problem gelöst?\\}
\pause
{\Huge Nein!}
\end{center}
\pause
\begin{alertblock}{Neues Problem:}
Jetzt können wir Namen, Semester und Matrikelnummer auch nicht mehr auslesen...
\end{alertblock}
\pause
\begin{block}{Auch hierzu gibt es aber eine Lösung:}
Mit \textbf{getter-Methoden} kann man den Lesezugriff auf Attribute wieder erlauben.
\end{block}
\end{frame}

\begin{frame}[containsverbatim]{getter-Methoden}
\begin{lstlisting}
public class Student {
//...Attribute, Konstruktor usw...
	
	//die getter-Methode fuer das Attribut 'name'
	public String getName() {
		return this.name;
	}
//...weitere getter-Methoden usw...
}
\end{lstlisting}
\begin{lstlisting}
public class Main {
	public static void main(String[] args) {
		Student maxMustermann = new Student("Max Mustermann", 3, 1234567);
		//liest den Namen und gibt ihn aus
		System.out.println(maxMustermann.getName());
		//...
\end{lstlisting}
\end{frame}

%%%%%%%%%%%%%%%%%%%%%%%%%%%%%%%%%%%%%%%%%%%%%%%%%%%%%%%%%%%%%%%%%%%%%%%%%
%%%%%%%%%%%%%%%%%%%%%%%%%%%%%%%%%%%%%%%%%%%%%%%%%%%%%%%%%%%%%%%%%%%%%%%%%

\subsection{Parameter und this-Schlüsselwort}
\begin{frame}[containsverbatim]
	\frametitle{Parameter-Bezeichnungen}
	
	\emph{Teddy.java}
	\begin{lstlisting}[language=java]
public class Teddy {
	String name;
	int groesse; //in cm

	public Teddy(String name, int groesse) {
		name = name;
		groesse = groesse;
	}
}
	\end{lstlisting}

	\emph{Main.java}
	\begin{lstlisting}[language=java]
Teddy myTeddy = new Teddy("Bruno", 30);
System.out.println(myTeddy.name);
	\end{lstlisting}
	
	Was wird ausgegeben?
	\pause
	\textbf{null} - Aber warum? %das ist invisible, siehe design.tex line 67
\end{frame}

%%%%%%%%%%%%%%%%%%%%%%%%%%%%%%%%%%%%%%%%%%%%%%%%%%%%%%%%%%%%%%%%%%%%%%%%%

\begin{frame}[containsverbatim]
	\frametitle{this - Schlüsselwort}
	
	\emph{Teddy.java}
	\begin{lstlisting}
public class Teddy {
	String name;
	int groesse; //in cm
			
	public Teddy(String name, int groesse) {
		this.name = name;
		this.groesse = groesse;
	}
}
	\end{lstlisting}

	\emph{Main.java}
	\begin{lstlisting}
Teddy myTeddy = new Teddy("Bruno", 30);
System.out.println(myTeddy.name);
	\end{lstlisting}
	
	
	Jetzt heißt unser Teddy wirklich Bruno :)
\end{frame}

%%%%%%%%%%%%%%%%%%%%%%%%%%%%%%%%%%%%%%%%%%%%%%%%%%%%%%%%%%%%%%%%%%%%%%%%%

\begin{frame}
	\frametitle{this - Schlüsselwort}
	
	Das \textbf{this}-Schüsselwort verweist auf das Objekt ''in dem'' man sich gerade befindet.
	
	$\rightarrow$ Das Objekt kann sich auf sich selbst beziehen.
\end{frame}

%%%%%%%%%%%%%%%%%%%%%%%%%%%%%%%%%%%%%%%%%%%%%%%%%%%%%%%%%%%%%%%%%%%%%%%%%
%%%%%%%%%%%%%%%%%%%%%%%%%%%%%%%%%%%%%%%%%%%%%%%%%%%%%%%%%%%%%%%%%%%%%%%%%
%%%%%%%%%%%%%%%%%%%%%%%%%%%%%%%%%%%%%%%%%%%%%%%%%%%%%%%%%%%%%%%%%%%%%%%%%

\section{Kontrollstrukturen}
\subsection{if-Anweisung}
\begin{frame}[containsverbatim]
	\frametitle{If-Anweisung}
	
		\emph{Syntax:}
		\begin{lstlisting}
if(<Bedingung>) {
	//Anweisungen fuer '<Bedingung> ist wahr'
} else {
	//Anweisungen fuer 'Bedingung ist falsch'
}
		\end{lstlisting}
		\emph{Erinnerung: Bedingungen sind Ausdrücke vom Typ 'boolean'}
\end{frame}

%%%%%%%%%%%%%%%%%%%%%%%%%%%%%%%%%%%%%%%%%%%%%%%%%%%%%%%%%%%%%%%%%%%%%%%%%

\begin{frame}[containsverbatim]
	\frametitle{If-Anweisung}
	
		\emph{Syntax:}
		\begin{lstlisting}
if(<Bedingung>) {
	//Anweisungen fuer 'Bedingung ist wahr'
} else if(<andere Bedingung>) {
	//Anweisungen fuer 'andere Bedingung ist wahr'
} else {
	//Anweisungen fuer 'Bedingung und andere Bedingung sind falsch'
}
		\end{lstlisting}
		\textbf{If-Anweisungen werden sehr oft gebraucht $\rightarrow$ merken!}
\end{frame}


%%%%%%%%%%%%%%%%%%%%%%%%%%%%%%%%%%%%%%%%%%%%%%%%%%%%%%%%%%%%%%%%%%%%%%%%%
%%%%%%%%%%%%%%%%%%%%%%%%%%%%%%%%%%%%%%%%%%%%%%%%%%%%%%%%%%%%%%%%%%%%%%%%%

\subsection{Switch-Anweisung}
\begin{frame}[containsverbatim]
	\frametitle{Switch-Anweisung}
	
		\emph{Syntax:}
		\begin{lstlisting}
switch (<Ausdruck>) {
	case <moeglicher Wert>:
		//Anweisungen, wenn '<Ausdruck>' == <moeglicher Wert>
		break;
	case <anderer moeglicher Wert>
		//...
		break;
	case default:
		//Anweisungen, wenn <Ausdruck> keinen angegebener Wert angenommen hat.
}\end{lstlisting}
		\textbf{Man beachte:}
		\begin{itemize}
			\item $<$moeglicher Wert$>$ muss eine Konstante sein
			\item $<$Ausdruck$>$ muss vom Typ char, byte, short, int oder String sein
			\item Vergessen von break kann gefährlich werden
		\end{itemize}
\end{frame}

%%%%%%%%%%%%%%%%%%%%%%%%%%%%%%%%%%%%%%%%%%%%%%%%%%%%%%%%%%%%%%%%%%%%%%%%%
%%%%%%%%%%%%%%%%%%%%%%%%%%%%%%%%%%%%%%%%%%%%%%%%%%%%%%%%%%%%%%%%%%%%%%%%%

\subsection{(do-)while-Schleifen}
\begin{frame}[containsverbatim]
	\frametitle{(do-)while-Schleifen }
	
	\emph{Syntax:}
	\begin{lstlisting}
while (<Bedingung>) {
	//Anweisungen werden ausgefuehrt, solange <Bedingung> == true;
}
		
//do-while Variante:
do {
	//Anweisungen werden ausgefuehrt, solange <Bedingung> == true;
} while(<Bedingung>);
	\end{lstlisting}
	\textbf{Wo ist der Unterschied?} \pause
	
	while-Schleifen prüft am Anfang,
	
	do-while-Schleife prüft am Ende $\rightarrow$ mindestens eine Ausführung des Rumpfes.
\end{frame}

%%%%%%%%%%%%%%%%%%%%%%%%%%%%%%%%%%%%%%%%%%%%%%%%%%%%%%%%%%%%%%%%%%%%%%%%%
%%%%%%%%%%%%%%%%%%%%%%%%%%%%%%%%%%%%%%%%%%%%%%%%%%%%%%%%%%%%%%%%%%%%%%%%%

\subsection{for-Schleife}
\begin{frame}[containsverbatim]
	\frametitle{for-schleife}
	
	\emph{Syntax:}
	\begin{lstlisting}
for(<Zaehler auf Start setzen>; <Bedingung mit Zaehler>; <Zaehler veraendern>) {
	//Rumpf wird so lange ausgefuehrt, bis <Bedingung mit Zaehler> falsch wird.
}
	\end{lstlisting}
	
\end{frame}
\begin{frame}[containsverbatim]
	\frametitle{for-schleife}	
	\emph{Typische Schleife:}
	\begin{lstlisting}
for(int i = 0; i <= 10; i++) {
	//Mach was sinnvolles.
}
	\end{lstlisting} \pause
	
	\textbf{Frage:} wie oft wird diese Schleife ausgeführt? \pause
	
	\textbf{Antwort:} 11 mal.
\end{frame}

%%%%%%%%%%%%%%%%%%%%%%%%%%%%%%%%%%%%%%%%%%%%%%%%%%%%%%%%%%%%%%%%%%%%%%%%%
%%%%%%%%%%%%%%%%%%%%%%%%%%%%%%%%%%%%%%%%%%%%%%%%%%%%%%%%%%%%%%%%%%%%%%%%%
%%%%%%%%%%%%%%%%%%%%%%%%%%%%%%%%%%%%%%%%%%%%%%%%%%%%%%%%%%%%%%%%%%%%%%%%%

\section{Aufgaben}
\subsection{Aufgaben} %Für das Design
\begin{frame}
	\frametitle{Aufgabenblatt}
\end{frame}

%%%%%%%%%%%%%%%%%%%%%%%%%%%%%%%%%%%%%%%%%%%%%%%%%%%%%%%%%%%%%%%%%%%%%%%%%
%%%%%%%%%%%%%%%%%%%%%%%%%%%%%%%%%%%%%%%%%%%%%%%%%%%%%%%%%%%%%%%%%%%%%%%%%
%%%%%%%%%%%%%%%%%%%%%%%%%%%%%%%%%%%%%%%%%%%%%%%%%%%%%%%%%%%%%%%%%%%%%%%%%

\section{Bonus}
\subsection{Bonus}
\begin{frame}[containsverbatim]
	\frametitle{Überlade von Methoden}
	
	\emph{HaushaltsRoboter.java}
	\begin{lstlisting}
public void geschirrAbraeumen(Geschirr was, Ort wohin) {
	//...
}
	\end{lstlisting}
	
	\emph{Program.java}
	\begin{lstlisting}
//...
HaushaltsRoboter robbi = new HaushaltsRoboter();
robbi.geschirrAbraeumen(myTelle, mySpuehlmaschiene);
robbi.geschirrAbraeumen(myTasse, mySpuehlmaschiene);
robbi.geschirrAbraeumen(myBesteck, mySpuehlmaschiene);
robbi.geschirrAbraeumen(myTopf, mySpuehlbecken);
//...
	\end{lstlisting}
	
	
	\textbf{Problem:} Wir geben bei fast allen Aufrufen der Methode den Parameter \emph{mySpuehlmaschiene} an.
\end{frame}

\begin{frame}[containsverbatim]
	\frametitle{Überlade von Methoden}
	
	\emph{HaushaltsRoboter.java}
	\begin{lstlisting}
public void geschirrAbraeumen(Geschirr was, Ort wohin) {
	//...
}
		
public void geschirrAbraeumen(Geschirr was) {
	//..
}
	\end{lstlisting}
	
	\emph{Program.java}
	\begin{lstlisting}
HaushaltsRoboter robbi = new HaushaltsRoboter();
robbi.geschirrAbraeumen(myTelle);
robbi.geschirrAbraeumen(myTasse);
robbi.geschirrAbraeumen(myBesteck);
robbi.geschirrAbraeumen(myTopf, mySpuehlbecken);
	\end{lstlisting}
	
	
	$\rightarrow$ \emph{geschirrAbrauemen} ist jetzt überladen.
\end{frame}

\begin{frame}
	\frametitle{Überladen von Methoden}
	
	So wird's gemacht:
	\begin{itemize}
		\item selber Rückgabetyp
		\item selber Name
		\item andere Parameter-Liste
	\end{itemize}
	
	
	Praktisch: Konstruktoren können auch überladen werden.
\end{frame}

\section{Zusammenfassung}
\subsection{Zusammenfassung} %Fürs Design
\begin{frame}
	\frametitle{Was haben wir heute gemacht?}
	
	\begin{itemize}
		\item \emph{this}-Schlüsselwort
		\item Ausdrücke / Anweisungen
		\item If- und Switch-Anweisung
		\item Drei verschiedene Schleifen
		\item (Überladen von Methoden)
	\end{itemize}
\end{frame}

%Noch fragen Folie
\section{Fragen?}
\subsection*{Fragen} %Für das Design...
\begin{frame}	
	\begin{center}
		\huge{Fragen?}
	\end{center}
\end{frame}



%comic
\begin{frame}[full]
\includegraphics[scale=0.55]{bilder/comics/September-25-2011-18-44-59-aa71ce1bd67502c27bc56a6b8d724897.jpeg}
\end{frame}
\end{document}

\end{document}