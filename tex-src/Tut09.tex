%%% Design basiert auf:
%% LaTeX-Beamer template for KIT design
%% by Erik Burger, Christian Hammer
%% title picture by Klaus Krogmann
%% version 2.1
\documentclass[18pt]{beamer}

\usepackage{templates/beamerthemekit}
\usepackage{graphicx} %because it will be included below
\usepackage{listings}
%\usepackage{wasysym}
\usepackage{color}
\usepackage[T1]{fontenc}
%\usepackage{cmap}
\usepackage{textcomp}
\usepackage[utf8]{inputenc}
%\usepackage{pdfpages}
\definecolor{listinggray}{gray}{0.9}
\definecolor{lbcolor}{rgb}{0.9,0.9,0.9}
\lstset{
        language=Java,
        backgroundcolor=\color{lbcolor},
        tabsize=4,
		keepspaces,
		extendedchars=true,
        rulecolor=\color{black},
        basicstyle=\footnotesize,
        aboveskip=0pt,
        upquote=true,
        columns=fixed,
        showstringspaces=false,
        extendedchars=true,
        breaklines=true,
        frame=single,
        showtabs=false,
        showspaces=false,
        showstringspaces=false,
        identifierstyle=\ttfamily,
        keywordstyle=\color[rgb]{0,0,1},
        commentstyle=\color[rgb]{0.133,0.545,0.133},
        stringstyle=\color[rgb]{0.627,0.126,0.941},
}


%% TITLE PICTURE
\titleimage{frontpic}


% For the title page
\title[Proggen WS11/12]{Programmieren WS 2011/2012}
\subtitle{Tutorium Nr. 1 / 11}
\author{Tobias Sturm} %, David Kulicke
\institute{Zertifizierbare Vertrauenswürdige Informatiksysteme}
\date[23.1.12] %TODO aktualisieren

% the presentation starts here
\begin{document}
\selectlanguage{ngerman}


%title page
\begin{frame}
	\titlepage
\end{frame}


%table of contents
\begin{frame}{Heute:}
%	\setcounter{tocdepth}{1}
	\tableofcontents
\end{frame}

\setbeamercovered{invisible}

%%%%%%%%%%%%%%%%%%%%%%%%%%%%%%%%%%%%%%%%%%%%%%%%%%%%%%%%%%%%%%%%%%%%%%%%%
%%%%%%%%%%%%%%%%%%%%%%%%%%%%%%%%%%%%%%%%%%%%%%%%%%%%%%%%%%%%%%%%%%%%%%%%%
%%%%%%%%%%%%%%%%%%%%%%%%%%%%%%%%%%%%%%%%%%%%%%%%%%%%%%%%%%%%%%%%%%%%%%%%%
\section*{Informationen}
\begin{frame}{Info}
	\begin{columns}[T]
		 \begin{column}{5cm}
			\includegraphics[scale=1.5]{bilder/eulenfest.png}		 
		 \end{column}
 		 \begin{column}{5cm}
			\textbf{Eulenfest}
			\begin{itemize}
				\item organisiert von Ersties
				\item hier im Infobau
				\item Tanzmatten und Zuckerwatte
				\item Helfer gesucht (v.A. Auf-/Abbau)
			\end{itemize}
 		 \end{column}
	\end{columns}
\end{frame}

\begin{frame}{Info}
	Im Praktomaten gibt es zwei 4. Übungsblätter!
	Gebt \textbf{entweder} beim ersten ab und schickt mir ne Mail, \textbf{oder} gebt beim späteren ab.
	

	Gebt bitte nur zur frühen Frist ab, wenn euch das Feedback von der Korrektur wichtig ist bzw. ihr mit eigenen Lösungen weiter rechnen wollt.
\end{frame}


%%%%%%%%%%%%%%%%%%%%%%%%%%%%%%%%%%%%%%%%%%%%%%%%%%%%%%%%%%%%%%%%%%%%%%%%%
%%%%%%%%%%%%%%%%%%%%%%%%%%%%%%%%%%%%%%%%%%%%%%%%%%%%%%%%%%%%%%%%%%%%%%%%%
%%%%%%%%%%%%%%%%%%%%%%%%%%%%%%%%%%%%%%%%%%%%%%%%%%%%%%%%%%%%%%%%%%%%%%%%%
\section{Übungsblatt 3}
\subsection{Häufig gemachte Fehler}
\begin{frame}{Übungsblatt 3 - Häufig gemachte Fehler}
	\begin{itemize}
		\item Javadoc fehlt
		\item JCC nicht eingehalten \pause
		\item Geheimnisprinzip verletzt \pause
		\item Compiler-Fehler \pause
		\item Sonderfälle nicht abgefangen
	\end{itemize}
\end{frame}

%%%%%%%%%%%%%%%%%%%%%%%%%%%%%%%%%%%%%%%%%%%%%%%%%%%%%%%%%%%%%%%%%%%%%%%%%

\begin{frame}[containsverbatim]
	\frametitle{Was ist hier falsch?}
	\begin{lstlisting}
public Profile[] getFriends() {
	Profile[] result = new Profile[getNumberOfFriends()];
	int i = 0;
		
	for (int j = 0; j < this.friends.length; j++) {
		if (this.friends[j] != null) {
			result[i] = this.friends[j];
		}
	}
}
	\end{lstlisting}
\end{frame}

%%%%%%%%%%%%%%%%%%%%%%%%%%%%%%%%%%%%%%%%%%%%%%%%%%%%%%%%%%%%%%%%%%%%%%%%%

\begin{frame}[containsverbatim]
	\frametitle{Übungsblatt 3}
	\begin{lstlisting}
public Profile[] getFriends() {
	Profile[] result = new Profile[getNumberOfFriends()];
	int i = 0;
		
	for (int j = 0; j < this.friends.length; j++) {
		if (this.friends[j] != null) {
			result[i] = this.friends[j];
			i++;
		}
	}
	return result;
}
	\end{lstlisting}
\end{frame}

%%%%%%%%%%%%%%%%%%%%%%%%%%%%%%%%%%%%%%%%%%%%%%%%%%%%%%%%%%%%%%%%%%%%%%%%%

\begin{frame}[containsverbatim]
	\frametitle{Übungsblatt 3}
	\begin{lstlisting}
public boolean addFriendship(Person p1, Person p2) {
	if (!containsPerson(p1) || !containsPerson(p2) || p1.isBefriendedWith(p1, p2) || p1 == null || p2 == null) {
			return false;
	}
	p1.addFriendship(p2);
	p2.addFriendship(p1);
	return true;
}
	\end{lstlisting}
\end{frame}

%%%%%%%%%%%%%%%%%%%%%%%%%%%%%%%%%%%%%%%%%%%%%%%%%%%%%%%%%%%%%%%%%%%%%%%%%

\begin{frame}{NullPointerException}
	\textbf{Was bedeutet das überhaupt?}\pause
	
	Aktion (Methode, Attributaufruf-/abfrage) auf eine \emph{null}-Referenz\pause
	
	\emph{null}-Referenzen sind Variabeln, deren Inhalt \emph{null} ist.\pause
	
	Exceptions treten erst bei der Ausführung des Programms auf und auch nicht immer!
\end{frame}

%%%%%%%%%%%%%%%%%%%%%%%%%%%%%%%%%%%%%%%%%%%%%%%%%%%%%%%%%%%%%%%%%%%%%%%%%

\begin{frame}[containsverbatim]
	\frametitle{NullPointerException}
	\begin{lstlisting}
public boolean addFriendship(Person p1, Person p2) {
	if(!p1.equals(null)) {
		//...
	}
}
	\end{lstlisting}
	Das geht schief, wenn p1 \emph{null} ist.
\end{frame}

%%%%%%%%%%%%%%%%%%%%%%%%%%%%%%%%%%%%%%%%%%%%%%%%%%%%%%%%%%%%%%%%%%%%%%%%%

\begin{frame}[containsverbatim]
	\frametitle{NullPointerException}
	\begin{lstlisting}
public boolean addFriendship(Person p1, Person p2) {
	if(p1 != null) {
		//...
	}
}
	\end{lstlisting}
	So prüfen wir, ob p1 eine \emph{null}-Referenz ist.
\end{frame}

%%%%%%%%%%%%%%%%%%%%%%%%%%%%%%%%%%%%%%%%%%%%%%%%%%%%%%%%%%%%%%%%%%%%%%%%%


\begin{frame}[containsverbatim]
	\frametitle{'Abkürzungen' bei if}
	\begin{lstlisting}
public boolean addFriendship(Person p1, Person p2) {
	if(p1 != null && !p1.isBefriendedWith(p2)) {
		//...
	}
}
	\end{lstlisting}
\end{frame}

%%%%%%%%%%%%%%%%%%%%%%%%%%%%%%%%%%%%%%%%%%%%%%%%%%%%%%%%%%%%%%%%%%%%%%%%%

\begin{frame}{Vergleichen}
	\textbf{Super-Wichtig!}
	
	\emph{equals} ist eine Methode, die Prüft, ob zwei Objekte die selben Eigenschaften haben
	
	\emph{==} prüft bei Referenz-Typen, ob es das identische Objekt ist!
\end{frame}

%%%%%%%%%%%%%%%%%%%%%%%%%%%%%%%%%%%%%%%%%%%%%%%%%%%%%%%%%%%%%%%%%%%%%%%%%
%%%%%%%%%%%%%%%%%%%%%%%%%%%%%%%%%%%%%%%%%%%%%%%%%%%%%%%%%%%%%%%%%%%%%%%%%
%%%%%%%%%%%%%%%%%%%%%%%%%%%%%%%%%%%%%%%%%%%%%%%%%%%%%%%%%%%%%%%%%%%%%%%%%
\section{Aufwärm-Aufgabe}
\subsection{Aufwärmen}
\begin{frame}{Aufwärm-Aufgabe}
	Welche Methoden werden mit welchen Parametern aufgerufen? (Aufruf-Baum)
	
	Was ist das Ergebnis des Programms?
\end{frame}

%%%%%%%%%%%%%%%%%%%%%%%%%%%%%%%%%%%%%%%%%%%%%%%%%%%%%%%%%%%%%%%%%%%%%%%%%

\begin{frame}{Methodenaufrufe}
	void main()\pause
	
	-- String getText()\pause
	
	-- -- String getTheName()\pause
	
	-- -- -- String getTheName('''') $\rightarrow$ Carlson\pause
	
	-- -- -- String getTheName(''Carlson '') $\rightarrow$ Carlson vom Dach\pause
	
	-- getDatum()\pause
	
	-- -- getDatum(''24.'')\pause
	
	-- -- -- getDatum(''24.12.'')\pause
	
	-- -- -- -- getDatum(''24.12..2011'') \pause
	
	-- -- -- -- -- getDatum(''24.12..2011 Weihnachten'') $\rightarrow$ 24.12..2011 Weihnachten\pause
	
	-- (wieder) String getText()$\rightarrow$ Ausgabe von allem.
\end{frame}

%%%%%%%%%%%%%%%%%%%%%%%%%%%%%%%%%%%%%%%%%%%%%%%%%%%%%%%%%%%%%%%%%%%%%%%%%
%%%%%%%%%%%%%%%%%%%%%%%%%%%%%%%%%%%%%%%%%%%%%%%%%%%%%%%%%%%%%%%%%%%%%%%%%
%%%%%%%%%%%%%%%%%%%%%%%%%%%%%%%%%%%%%%%%%%%%%%%%%%%%%%%%%%%%%%%%%%%%%%%%%

\section{Rekursion}
\subsection*{Was, wozu und wie?}
\begin{frame}[containsverbatim]
	\frametitle{Rekursion}
	\textbf{Rekursion nimmt einen Arbeit ab}	
	
	Eine Methode arbeitet rekursiv, wenn sie sich selbst aufruft.
	
	\begin{lstlisting}
public int methode(int p) {
	//...
	return methode(p + x);
}
	\end{lstlisting}
\end{frame}

%%%%%%%%%%%%%%%%%%%%%%%%%%%%%%%%%%%%%%%%%%%%%%%%%%%%%%%%%%%%%%%%%%%%%%%%%

\begin{frame}{Rekursion}
	Was braucht man für eine Abbruchbedingung?
	
	\begin{itemize}
		\item eine Methode \pause
		\item eine Abbruchbedingung \pause
		\item Wertveränderung für rekursiven Aufruf \pause
	\end{itemize}
	
	\textbf{Abbruchbedingung} weil wir sonst in einer Endlosschleife sind.
	\textbf{Wertveränderung} weil wir sonst nichts gewonnen haben.
\end{frame}

%%%%%%%%%%%%%%%%%%%%%%%%%%%%%%%%%%%%%%%%%%%%%%%%%%%%%%%%%%%%%%%%%%%%%%%%%

\section{Aufgaben}
\subsection{Rekursion}
\begin{frame}{Aufgabe - Fakultät berechnen}
	Schreibe eine Methode, die $n!$ berechnet.
	
	\textbf{Tips:}
	\begin{itemize}
		\item Rekursiv lösen
		\item Als erstes Abbruchbedingung überlegen
		\item Parameter < 1 kommen einfach nicht vor.
		\item benutz die Datei \emph{FacCalc.java}
		\item Versucht es ohne Eclipse ;-P
	\end{itemize}
\end{frame}

%%%%%%%%%%%%%%%%%%%%%%%%%%%%%%%%%%%%%%%%%%%%%%%%%%%%%%%%%%%%%%%%%%%%%%%%%


\begin{frame}{Aufgabe - Fibonacci-Folge}
	Schreibe eine Methode, die Fibonacci-Folgen berechnen kann
	
	$f(n) = \begin{Bmatrix}
		n & n = 1 oder 0 \\
		f(n - 1) + f(n - 2) & sonst
	\end{Bmatrix}$
	
	\textbf{Tips:}
	\begin{itemize}
		\item Rekursiv lösen
		\item Als erstes Abbruchbedingung überlegen
		\item Parameter < 0 kommen einfach nicht vor.
		\item benutz die Datei \emph{FiboCalc.java}
		\item Versucht es ohne Eclipse ;-P
	\end{itemize}
\end{frame}

%%%%%%%%%%%%%%%%%%%%%%%%%%%%%%%%%%%%%%%%%%%%%%%%%%%%%%%%%%%%%%%%%%%%%%%%%


\begin{frame}{Aufgabe - Bonus-Fibonacci-Folge}
	Erweitere die Lösung, sodass bereits berechnete Ergebnisse wiederverwendet werden können.
	
	\textbf{Tips:}
	\begin{itemize}
		\item Array
	\end{itemize}
\end{frame}

%%%%%%%%%%%%%%%%%%%%%%%%%%%%%%%%%%%%%%%%%%%%%%%%%%%%%%%%%%%%%%%%%%%%%%%%%
%%%%%%%%%%%%%%%%%%%%%%%%%%%%%%%%%%%%%%%%%%%%%%%%%%%%%%%%%%%%%%%%%%%%%%%%%
%%%%%%%%%%%%%%%%%%%%%%%%%%%%%%%%%%%%%%%%%%%%%%%%%%%%%%%%%%%%%%%%%%%%%%%%%

\section{Übungsblatt 4}
\subsection*{Hinweise}
\begin{frame}{zum aktuellen ÜB4}

	Zu Links gab es sehr viele unterschiedliche Ansätze. Übernehmt bitte die Musterlösung (auch wenn sie \emph{doof} ist)!
	
	\begin{itemize}
		\item schmeißt \emph{KITBook} / \emph{KITBook} aus \emph{KITBook} raus!
		\item Verwendet die Musterlösung
		\item Wurzeln berechnen mit \emph{Math.sqrt(double)}
		\item Wenn man nicht weiter weiß: Signatur hin schreiben und weiter mit nächster Aufgabe
		\item Macht die Zusatzaufgabe
	\end{itemize}
\end{frame}

%Noch fragen Folie
\section{Fragen?}
\subsection*{Fragen} %Für das Design...
\begin{frame}	
	\begin{center}
		\huge{Fragen?}
	\end{center}
\end{frame}



%comic
\begin{frame}[full]
\includegraphics[scale=0.55]{bilder/comics/September-25-2011-18-44-59-aa71ce1bd67502c27bc56a6b8d724897.jpeg}
\end{frame}
\end{document}

\end{document}