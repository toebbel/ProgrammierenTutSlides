%%% Design basiert auf:
%% LaTeX-Beamer template for KIT design
%% by Erik Burger, Christian Hammer
%% title picture by Klaus Krogmann
%% version 2.1
\documentclass[18pt]{beamer}

\usepackage{templates/beamerthemekit}
\usepackage{graphicx} %because it will be included below
\usepackage{listings}
%\usepackage{wasysym}
\usepackage{color}
\usepackage[T1]{fontenc}
%\usepackage{cmap}
\usepackage{textcomp}
\usepackage[utf8]{inputenc}
%\usepackage{pdfpages}
\definecolor{listinggray}{gray}{0.9}
\definecolor{lbcolor}{rgb}{0.9,0.9,0.9}
\lstset{
        language=Java,
        backgroundcolor=\color{lbcolor},
        tabsize=4,
		keepspaces,
		extendedchars=true,
        rulecolor=\color{black},
        basicstyle=\footnotesize,
        aboveskip=0pt,
        upquote=true,
        columns=fixed,
        showstringspaces=false,
        extendedchars=true,
        breaklines=true,
        frame=single,
        showtabs=false,
        showspaces=false,
        showstringspaces=false,
        identifierstyle=\ttfamily,
        keywordstyle=\color[rgb]{0,0,1},
        commentstyle=\color[rgb]{0.133,0.545,0.133},
        stringstyle=\color[rgb]{0.627,0.126,0.941},
}


%% TITLE PICTURE
\titleimage{frontpic}


% For the title page
\title[Proggen WS11/12]{Programmieren WS 2011/2012}
\subtitle{Tutorium Nr. 1 / 11}
\author{Tobias Sturm} %, David Kulicke
\institute{Zertifizierbare Vertrauenswürdige Informatiksysteme}
\date[23.1.12] %TODO aktualisieren

% the presentation starts here
\begin{document}
\selectlanguage{ngerman}


%title page
\begin{frame}
	\titlepage
\end{frame}


%table of contents
\begin{frame}{Heute:}
%	\setcounter{tocdepth}{1}
	\tableofcontents
\end{frame}

\setbeamercovered{invisible}

%%%%%%%%%%%%%%%%%%%%%%%%%%%%%%%%%%%%%%%%%%%%%%%%%%%%%%%%%%%%%%%%%%%%%%%%%
%%%%%%%%%%%%%%%%%%%%%%%%%%%%%%%%%%%%%%%%%%%%%%%%%%%%%%%%%%%%%%%%%%%%%%%%%
%%%%%%%%%%%%%%%%%%%%%%%%%%%%%%%%%%%%%%%%%%%%%%%%%%%%%%%%%%%%%%%%%%%%%%%%%
\section{Wissenswertes}
\subsection{Checkstyle}
\begin{frame}{Checkstyle installieren}
	Menü 'Help' $\rightarrow$ install new Software $\rightarrow$ Button 'add' $\rightarrow$ beliebeigen Namen eingeben und 'http://eclipse-cs.sf.net/update/' als URL. 'OK' $\rightarrow$ Eintrag auswählen $\rightarrow$ Warten. Dann 'Checkstyle' 'ausklappen' $\rightarrow$ Haken bei 'Eclipse Checkstyle Plugin' $\rightarrow$ 'Next' $\rightarrow$ AGBs bli bla blub. Fertig!
\end{frame}

%%%%%%%%%%%%%%%%%%%%%%%%%%%%%%%%%%%%%%%%%%%%%%%%%%%%%%%%%%%%%%%%%%%%%%%%%

\begin{frame}{Checkstyle einrichten}
	Im Menü von Eclipse 'Windows' $\rightarrow$ 'Preferences' $\rightarrow$ Punkt 'Checkstyle' $\rightarrow$ Button 'New' $\rightarrow$ Type 'external configuration file', Name z.B. ProggenCheckStyle, mit 'Location' die heruntergeladene Checkstyle-Datei der Vorlesung auswählen $\rightarrow$ 'Ok' $\rightarrow$ in Liste neuen Eintrag auswählen $\rightarrow$ 'Set as Default' $\rightarrow$ 'Ok'
\end{frame}

%%%%%%%%%%%%%%%%%%%%%%%%%%%%%%%%%%%%%%%%%%%%%%%%%%%%%%%%%%%%%%%%%%%%%%%%%

\begin{frame}{Checkstyle verwenden}
	Rechtsklick auf src-Ordner im Package Explorer $\rightarrow$ 2. Eintrag von unten 'Checkstyle' $\rightarrow$ 'check with Checkstyle. $\rightarrow$ Checkstyle-Sachen tauchen bei den Problemen auf.
\end{frame}

%%%%%%%%%%%%%%%%%%%%%%%%%%%%%%%%%%%%%%%%%%%%%%%%%%%%%%%%%%%%%%%%%%%%%%%%%
%%%%%%%%%%%%%%%%%%%%%%%%%%%%%%%%%%%%%%%%%%%%%%%%%%%%%%%%%%%%%%%%%%%%%%%%%
\subsection{Terminal}
\begin{frame}{Terminal}
	
	\begin{itemize}
		\item von der VL-Seite herunterladen (neben dem ÜB).
		\item in den selben Ordner wie andere Quelltext-Dateien
		\item Übernimmt für euch Arbeit :)
	\end{itemize}
	
\end{frame}

%%%%%%%%%%%%%%%%%%%%%%%%%%%%%%%%%%%%%%%%%%%%%%%%%%%%%%%%%%%%%%%%%%%%%%%%%

\begin{frame}[containsverbatim]
	\frametitle{Terminal}
	\emph{Main.java}
	\begin{lstlisting}
public static void main(String[] args) {
	String input;
	input = Terminal.askString("Gib einen Satz ein");
	System.out.println("Du hast '" + input + "' eingegeben");
}
	\end{lstlisting}
\end{frame}

%%%%%%%%%%%%%%%%%%%%%%%%%%%%%%%%%%%%%%%%%%%%%%%%%%%%%%%%%%%%%%%%%%%%%%%%%

\begin{frame}{Strings in wörter zerlegen}
	$\rightarrow$ SplitDemo.java
\end{frame}

%%%%%%%%%%%%%%%%%%%%%%%%%%%%%%%%%%%%%%%%%%%%%%%%%%%%%%%%%%%%%%%%%%%%%%%%%

\begin{frame}{Groß- und Kleinschreibung}
	\textbf{toLowerCase()}

	\begin{itemize}
		\item wird auf Strings ausgeführt
		\item gibt wieder einen String zurück
	\end{itemize}
\end{frame}

%%%%%%%%%%%%%%%%%%%%%%%%%%%%%%%%%%%%%%%%%%%%%%%%%%%%%%%%%%%%%%%%%%%%%%%%%

\begin{frame}[containsverbatim]
	\frametitle{{Aufgabe}}
	\textbf{Aufgabe:} Verändere das Programm so, dass die Eingabe wortweise umgedreht wird. Alle Wörter sollen klein geschrieben werden.
	\begin{lstlisting}
public static void main(String[] args) {
	String input;
	input = Terminal.askString("Gib einen Satz ein");
	System.out.println("Du hast '" + input + "' eingegeben");
}
	\end{lstlisting}
\end{frame}

\section{Aufgabenblatt}
\subsection{Aufgabenblatt}
\begin{frame}{Aufgabenblatt}
\end{frame}

%%%%%%%%%%%%%%%%%%%%%%%%%%%%%%%%%%%%%%%%%%%%%%%%%%%%%%%%%%%%%%%%%%%%%%%%%
%%%%%%%%%%%%%%%%%%%%%%%%%%%%%%%%%%%%%%%%%%%%%%%%%%%%%%%%%%%%%%%%%%%%%%%%%
%%%%%%%%%%%%%%%%%%%%%%%%%%%%%%%%%%%%%%%%%%%%%%%%%%%%%%%%%%%%%%%%%%%%%%%%%
\section{Übungsblatt 3}
\subsection{Worauf man achten muss}
\begin{frame}{Übungsblatt 3}
	\textbf{Tips}
	
	\begin{itemize}
		\item Benutzt die Musterlösung von ÜB2 \pause
		\item Schreibt Quelltext und Kommentare auf Englisch \pause
		\item Benutzt Ecplipse + CheckStyle-Plugin \pause
		\item Schreibt Javadoc sofort, \pause
		\item \textbf{Fang heute an!}
	\end{itemize}

\end{frame}

\begin{frame}{Übungsblatt 3}
	\textbf{Java API}
	
	Man darf \emph{java.util.} und alle Subpakete \textbf{nicht} verwenden. Ziel der Übung ist selbst eine Implementierung zu schreiben!

\end{frame}

\begin{frame}{Übungsblatt 3}
	\textbf{Freiheiten / Restriktionen}
	
	\begin{itemize}
		\item Hilfsmethoden gut und erwünscht $\rightarrow$ private
		\item Keine eigenen public-Methoden
		\item Geforderte Signaturen \textbf{nicht} verändern
	\end{itemize}
\end{frame}


%Noch fragen Folie
\section{Fragen?}
\subsection*{Fragen} %Für das Design...
\begin{frame}	
	\begin{center}
		\huge{Fragen?}
	\end{center}
\end{frame}



%comic
\begin{frame}[full]
\includegraphics[scale=0.55]{bilder/comics/September-25-2011-18-44-59-aa71ce1bd67502c27bc56a6b8d724897.jpeg}
\end{frame}
\end{document}

\end{document}