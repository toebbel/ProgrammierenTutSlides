%%% Design basiert auf:
%% LaTeX-Beamer template for KIT design
%% by Erik Burger, Christian Hammer
%% title picture by Klaus Krogmann
%% version 2.1
\documentclass[18pt]{beamer}

\usepackage{templates/beamerthemekit}
\usepackage{graphicx} %because it will be included below
\usepackage{listings}
%\usepackage{wasysym}
\usepackage{color}
\usepackage[T1]{fontenc}
%\usepackage{cmap}
\usepackage{textcomp}
\usepackage[utf8]{inputenc}
%\usepackage{pdfpages}
\definecolor{listinggray}{gray}{0.9}
\definecolor{lbcolor}{rgb}{0.9,0.9,0.9}
\lstset{
        language=Java,
        backgroundcolor=\color{lbcolor},
        tabsize=4,
		keepspaces,
		extendedchars=true,
        rulecolor=\color{black},
        basicstyle=\footnotesize,
        aboveskip=0pt,
        upquote=true,
        columns=fixed,
        showstringspaces=false,
        extendedchars=true,
        breaklines=true,
        frame=single,
        showtabs=false,
        showspaces=false,
        showstringspaces=false,
        identifierstyle=\ttfamily,
        keywordstyle=\color[rgb]{0,0,1},
        commentstyle=\color[rgb]{0.133,0.545,0.133},
        stringstyle=\color[rgb]{0.627,0.126,0.941},
}


%% TITLE PICTURE
\titleimage{frontpic}


% For the title page
\title[Proggen WS11/12]{Programmieren WS 2011/2012}
\subtitle{Tutorium Nr. 1 / 11}
\author{Tobias Sturm} %, David Kulicke
\institute{Zertifizierbare Vertrauenswürdige Informatiksysteme}
\date[23.1.12] %TODO aktualisieren

% the presentation starts here
\begin{document}
\selectlanguage{ngerman}


%title page
\begin{frame}
	\titlepage
\end{frame}


%table of contents
\begin{frame}{Heute:}
%	\setcounter{tocdepth}{1}
	\tableofcontents
\end{frame}

\setbeamercovered{invisible}

\section{Datentypen}
\subsection{primitiv vs. Referenz}

\begin{frame}
	\frametitle{Primitive und nicht primitive}
	
	\begin{itemize}
		\item<1-> In Java gibt es 8 primitive Datentypen. Der Rest sind Klassen.
		\item<2-> Primitive Datentypen brauchen kein \emph{new} bei 'Objekterzeugung'.
		\item<3-> Primitive Datentypen $\neq$ ungleich Klassen/Objekt
	\end{itemize}
\end{frame}

%%%%%%%%%%%%%%%%%%%%%%%%%%%%%%%%%%%%%%%%%%%%%%%%%%%%

\begin{frame}
	\frametitle{Primitive Datentypen}
	\textbf{Wir brauchen (fast) nur diese primitiven Datentypen:}
	\begin{itemize}
		\item int
		\item boolean
		\item char
		\item double
	\end{itemize}
	
	\pause
	Und was ist mit String?
\end{frame}

%%%%%%%%%%%%%%%%%%%%%%%%%%%%%%%%%%%%%%%%%%%%%%%%%%%%

\begin{frame}[fragile]
	\frametitle{String}
	
	String ist kein primitiver Datentyp!
	Trotzdem braucht man kein \emph{new}.
	
	\begin{lstlisting}[language=java]
		String myVar = "Hallo";		//String mit Inhalt
		String myVar2 = new String(); 	//leerer String
		String myVar3 = ""; 			//auch leerer String
	\end{lstlisting}
\end{frame}

%%%%%%%%%%%%%%%%%%%%%%%%%%%%%%%%%%%%%%%%%%%%%%%%%%%%
\subsection{Umgang mit primitven Datentypen}
\begin{frame}[fragile]
	\frametitle{String}
	
	String ist kein primitiver Datentyp!
	Trotzdem kann man "rechnen"
	
	\begin{lstlisting}[language=java]
		String myVar = "Hallo";		
		String myVar2 = "Welt"; 
		
		//Verknuepfen von Strings
		String result1 = myVar + " " + myVar2;
		String result2 = myVar2 + "+" + myVar2;
	\end{lstlisting}
\end{frame}

%%%%%%%%%%%%%%%%%%%%%%%%%%%%%%%%%%%%%%%%%%%%%%%%%%%%

\begin{frame}[fragile]
	\frametitle{String und Escape-Zeichen}
	
	String mit Inhalt \emph{Hallo ''Welt''}.
	\begin{lstlisting}[language=java]
		String = "Hallo \"Welt\"";
	\end{lstlisting}
	\pause
	
	
	String mit Zeilenumbruch in der Mitte.
	\begin{lstlisting}[language=java]
		String = "Hallo \n Welt";
	\end{lstlisting}
	\pause
	
	
	String mit Inhalt \emph{Hallo \textbackslash Welt}.
	\begin{lstlisting}[language=java]
		String = "Hallo \\Welt";
	\end{lstlisting}
\end{frame}

%%%%%%%%%%%%%%%%%%%%%%%%%%%%%%%%%%%%%%%%%%%%%%%%%%%%

\begin{frame}[fragile]
	\frametitle{Char}
	
	Immer in einfachen Anführungszeichen
	\begin{lstlisting}[language=java]
		char einA = 'A';
		char NochEinA = 65;
	\end{lstlisting}
	Was passiert bei 
	\begin{lstlisting}[language=java]
	einA + NochEinA
	\end{lstlisting}	
\end{frame}

%%%%%%%%%%%%%%%%%%%%%%%%%%%%%%%%%%%%%%%%%%%%%%%%%%%%

\begin{frame}[fragile]
	\frametitle{Double / Float}
	
	\textbf{double}
	Kein , sondern . verwenden! Für Floats immer ein \emph{f} anhängen.
	\begin{lstlisting}[language=java]
		double myDouble = 123.456;
		float myFloat = 9.81f;
	\end{lstlisting}
\end{frame}

%%%%%%%%%%%%%%%%%%%%%%%%%%%%%%%%%%%%%%%%%%%%%%%%%%%%

\begin{frame}[fragile]
	\frametitle{Zahlendarstellungen}
	
	Man kann Zahlen unterschiedlich darstellen:
	\begin{lstlisting}[language=java]
		int a = 18; //Dezimal
		int a = 022; //Oktal
		int a = 0x12; //Hexadezimal
	\end{lstlisting}
	alle Zeilen bedeuten das selbe!
\end{frame}

%%%%%%%%%%%%%%%%%%%%%%%%%%%%%%%%%%%%%%%%%%%%%%%%%%%%
\subsection{Welcher Datentyp für was?}
\begin{frame}[fragile]
	\frametitle{Übersicht Datentypen}
	\begin{tabular}[ht]{|l|l|l|}
		\hline
		\textbf{Typ}&	\textbf{Erklärung}		&	\textbf{Beispiel-Werte}								 \\
		\hline
		boolean 	&	Wahrheitswerte 			&	$true$, $false$										 \\
		char 		&	16-Bit-Unicode-Zeichen 	&	$’A’$, ’\textbackslash $n$', ’\textbackslash $u05D0$’	 \\
		byte 		&	8-Bit-Ganzzahl			& 	$12$												 \\
		short 		&	16-Bit-Ganzzahl 		&	$12$												 \\
		int 		&	32-Bit-Ganzzahl			& 	$12$												 \\
		long 		&	64-Bit-Ganzzahl			& 	$12L$, $14l$										 \\
		float 		&	32-Bit-Gleitpunktzahlen &	$9.81F$, $0.379E-8F$, $2f$							 \\
		double 		&	64-Bit-Gleitpunktzahlen &	$9.81$, $0.379E-8$, $3e1$								 \\
		\hline
	\end{tabular}

\end{frame}

%%%%%%%%%%%%%%%%%%%%%%%%%%%%%%%%%%%%%%%%%%%%%%%%%%%%
\begin{frame}[fragile]
	\frametitle{Wertebereiche}
	\begin{tabular}[ht]{|l|l|l|}
		\hline
		\textbf{Typ}	&	\textbf{kleinster Wert}			&	\textbf{größter Wert} 			\\
		\hline
		char 			&	\textbackslash u0000 (0)		&	\textbackslash uFFFF (65.535)	\\
		byte 			&	$-128$							&	$127$								\\
		short 			&	$-32.768$						&	$32.767$							\\
		int				&	$-2.147.483.648$				&	$2.147.483.647$					\\
		long 			&	$-9.223.372.036.854.775.808$	&	$9.223.372.036.854.775.807$		\\
		float			&	$-3.4028235*10^{38}$				&	$3.4028235*10^{38}$					\\
		double 			&	$-1.7976931348623157*10^{308}$		&	$1.7976931348623157*10^{308}$		\\
		\hline
	\end{tabular}

\end{frame}

%%%%%%%%%%%%%%%%%%%%%%%%%%%%%%%%%%%%%%%%%%%%%%%%%%%%

\section{Operatoren}
\subsection{Zuweisungen}
\begin{frame}[fragile]
	\frametitle{Zuweisungsoperator}
	
	Anders als bei Mathe:
	
	\begin{lstlisting}[language=java]
		int x;		//Deklarieren
		int y = 3; 	//Deklarieren und initialisieren
		y = y + 3; //das ist KEINE Gleichung
		x = y;
	\end{lstlisting}
\end{frame}

%%%%%%%%%%%%%%%%%%%%%%%%%%%%%%%%%%%%%%%%%%%%%%%%%%%%

\subsection{Zahlen}
\begin{frame}
	\frametitle{Was wir brauchen}
	\framesubtitle{Operatoren für Zahlen}
	
	\begin{tabular}[ht]{|l|c|c|}
		\hline
								& \textbf{in Java}		&	\textbf{Ergebnis}	\\
		\hline
		Zuweisung 				& = 					&	wahr				\\ %%TODO ist das so?
		\hline
		add, sub, mult 			& $+$, $-$, $*$ 		&	genauster Datentyp	\\
		\hline
		div, mod 				& $/$, $\%$			 	&	Zahl (abhängig)		\\ %%TODO ausprobieren
		\hline
		Vergleich 				& $<$, $<=$, $>=$, $>$ 	&	bool'scher Wert		\\
		\hline
		(Un-)Gleich			 	& $==$, $!=$ 			&	bool'scher Wert		\\
		\hline
	\end{tabular}
\end{frame}


%%%%%%%%%%%%%%%%%%%%%%%%%%%%%%%%%%%%%%%%%%%%%%%%%%%%

\subsection{logisches und bitweises Zeug}
\begin{frame}
	\frametitle{Logische / Bitweise Operatoren}
	\framesubtitle{Operanden für boolsche Werte}
	
	Verknüpfung von bool'schen Werten
	
	
	\begin{tabular}[ht]{|l|c|}
		\hline
		logisches Nicht & $!$ \\
		\hline
		logisches Und & $\&\&$ \\
		\hline
		logisches Oder & $||$ \\
		\hline
		bitweises Nicht / Komplement & \textasciitilde \\
		\hline
		bitweises Oder & $|$ \\
		\hline
		bitweises Und & $\&$ \\
		\hline
		exklusives Oder & \^{} \\
		\hline
	\end{tabular}
\end{frame}

%%%%%%%%%%%%%%%%%%%%%%%%%%%%%%%%%%%%%%%%%%%%%%%%%%%%

\subsection{Interpretationssache}
\begin{frame}
	\frametitle{Präzedenzen von Operatoren}
	\begin{enumerate}
		\item Klammerung \pause
		\item Präzedenz	\pause
		\item Links nach Rechts
	\end{enumerate}
	
\end{frame}


%%%%%%%%%%%%%%%%%%%%%%%%%%%%%%%%%%%%%%%%%%%%%%%%%%%%

\section{Variabeln und Attribute}
\subsection{Begriffe}
\begin{frame}
	\frametitle{Variabeln}
	Was gehört zusammen?
	
	
	\begin{columns}[c]
		\column[c]{5cm}
			\begin{enumerate}
				\item Initialisieren und Deklarieren
				\item Deklarieren
				\item Zuweisung
			\end{enumerate}
		\column{5cm}
			\begin{enumerate}
				\item int i;
				\item int i = 0;
				\item i = 0;
			\end{enumerate}
	\end{columns}
	
	
	\pause
	$\Rightarrow$ Attribute sind Variabeln die auf Klassenebene deklariert werden.
\end{frame}

%%%%%%%%%%%%%%%%%%%%%%%%%%%%%%%%%%%%%%%%%%%%%%%%%%%%

\subsection{Auf Attribute zugreifen}
\begin{frame}[fragile]
	\frametitle{Attributzugriff}
	
	\begin{lstlisting}[language=java]
		public class Auto {
			public String farbe;	//in Deutscher Sprache
			public int baujahr;		//Format YYYY
			public double gewicht; //in kg
		}
	\end{lstlisting}
	
	\begin{lstlisting}[language=java]
		Auto myAuto = new Auto(); //deklarieren und initialisieren mit neuem Objekt
		
		myAuto.farbe = "blau";
		myAuto.baujahr = 1985;
		myAuto.gewicht = 987.2;
	\end{lstlisting}
	
\end{frame}

%%%%%%%%%%%%%%%%%%%%%%%%%%%%%%%%%%%%%%%%%%%%%%%%%%%%
\begin{frame}[fragile]
	\frametitle{Attributzugriff}
	Was ist hier falsch?
	
	
	\begin{lstlisting}[language=java]
		public class Auto {
			public String farbe;	//in Deutscher Sprache
			public int baujahr;		//Format YYYY
			public double gewicht; //in kg
		}
	\end{lstlisting}
	
	\begin{lstlisting}[language=java]
		Auto.farbe = "rot";?
	\end{lstlisting}
	
\end{frame}

%%%%%%%%%%%%%%%%%%%%%%%%%%%%%%%%%%%%%%%%%%%%%%%%%%%%
\subsection{Konstanten}
\begin{frame}
	\frametitle{Konstanten - Wozu braucht man das?}
\end{frame}

%%%%%%%%%%%%%%%%%%%%%%%%%%%%%%%%%%%%%%%%%%%%%%%%%%%%

\begin{frame}[fragile]
	\frametitle{Konstanten - Wozu braucht man das?}
	
	\begin{lstlisting}[language=java]
		public class FlaechenBerechner {
		
			private final double pi = 3.1415926535897932;
		
			public double kreisFlaeche(double radius) {
				...
			}
			
			public double kugelVolumen(double radius) {
				...
			}
		}
	\end{lstlisting}
\end{frame}

%%%%%%%%%%%%%%%%%%%%%%%%%%%%%%%%%%%%%%%%%%%%%%%%%%%%

\begin{frame}[fragile]
	\frametitle{Konstanten - Wozu braucht man das?}
	
	\begin{lstlisting}[language=java]
		public class FileHelper {
			private final String rootDir = "C:\\folder\\boing\\";
			
			public void openFile(String relPath)
			{
				...
			}
		}
	\end{lstlisting}
\end{frame}

%%%%%%%%%%%%%%%%%%%%%%%%%%%%%%%%%%%%%%%%%%%%%%%%%%%%

\subsection{Klassen-Variabeln}
\begin{frame}[fragile]
	\frametitle{Klassen-Variabeln}
	
	\begin{lstlisting}[language=java]
		public class Jeep {
			public static int ANZAHLRAEDER = 4;
			...
		}
	\end{lstlisting}
\end{frame}
	
\begin{frame}[fragile]
	\frametitle{Klassen-Variabeln}
	Zugriff über:
	\begin{lstlisting}[language=java]
		int var = Jeep.ANZAHLRAEDER; //Zugriff wie er sein sollte
		
		Jeep myJeep = new Jeep(); //Objekterzeugung
		int var2 = myJeep.ANZAHLRAEDER; //Zugriff wie er nicht sein sollte, aber geht.
	\end{lstlisting}
	
	
	Anzahl der Räder für \textbf{jeden} Jeep, auch für die, die nicht ''gebaut'' wurden.
\end{frame}

%%%%%%%%%%%%%%%%%%%%%%%%%%%%%%%%%%%%%%%%%%%%%%%%%%%%

\section{Aufgabenblatt}
\subsection{Aufgabenblatt}
\begin{frame}
	\frametitle{Aufgabenblatt}
\end{frame}

%%%%%%%%%%%%%%%%%%%%%%%%%%%%%%%%%%%%%%%%%%%%%%%%%%%%

\section{Referenzen auf Objekte}
\subsection{Referenzen}
\begin{frame}[fragile]
	\frametitle{Referenzen - was bedeuted das?}
	Auto.java:
	\begin{lstlisting}[language=java]
		public class Auto {
			public String farbe;
		}
	\end{lstlisting}
	
	main.java:
	\begin{lstlisting}[language=java]
		Auto auto1 = new Auto();
		Auto auto2 = new Auto();
		Auto auto3 = auto1;
		
		auto1.farbe = "braun";
		auto2.farbe = "gelb";
		auto3.farbe = "blau";
	\end{lstlisting}
	Welches Auto hat welche Farbe?
	
	\pause	auto1: blau, auto2: gelb, auto3: blau.
	
	Warum ist das so?
	
\end{frame}

%%%%%%%%%%%%%%%%%%%%%%%%%%%%%%%%%%%%%%%%%%%%%%%%%%%%

\begin{frame}[fragile]
	\frametitle{Referenzen - zu beachten}
	
	Zuweisungen von Objekten bewirkt eine Kopie der Referenz.
	
	
	\begin{lstlisting}[language=java]
		Teddybaer timmy = new Teddybaer();
		Teddybear tilo = timmy;
	\end{lstlisting}
	
	
	Timmy und Tilo sind \textbf{ein und der selbe Teddybaer}! Es gibt eben zwei verschiedene Namen für diesen einen Bären.
\end{frame}

%%%%%%%%%%%%%%%%%%%%%%%%%%%%%%%%%%%%%%%%%%%%%%%%%%%%

\section{Methoden}
\subsection{Was ist eine Methode}
\begin{frame}[fragile]
	\frametitle{Mach mal! - die Methode}
	
	Wir wollen mit Objekten etwas machen. Bzw. Objekte sollen etwas machen.
	
	
	\pause
	Taschenrechner.java:
	\begin{lstlisting}[language=java]
		public class Taschenrechner {
			public int addiere(int a, int b) {
				int ergebnis = a + b;
				return ergebnis
			}
		}
	\end{lstlisting}
	\pause
	
	
	main.java:
	\begin{lstlisting}[language=java]
		Taschenrechner myTR = new Taschenrechner();
		int wasIchWissenWill = myTR.addiere(1, 1);
	\end{lstlisting}
\end{frame}

%%%%%%%%%%%%%%%%%%%%%%%%%%%%%%%%%%%%%%%%%%%%%%%%%%%%

\subsection{Methoden ohne Rückgabewert}
\begin{frame}[fragile]
	\frametitle{Methoden die nur nehmen}
	Taschenrechner.java:
	\begin{lstlisting}[language=java]
		public class Taschenrechner {
			int speicher;
			
			public void speichereZahl(int zahl) {
				speicher = zahl;
			}
		}
	\end{lstlisting}
\end{frame}

%%%%%%%%%%%%%%%%%%%%%%%%%%%%%%%%%%%%%%%%%%%%%%%%%%%%

\subsection{Übung}
\begin{frame}[fragile]
	\frametitle{Was ist was?}
	Autopilot.java:
	\begin{lstlisting}[language=java]
		public class Autopilot {
			
			public void reset() {
				//hier wird alles zurueck gesetzt
			}
			
			public int getAltitude() {
				...
				return altitude;
			}
			
			public void setDesiredAltitude(int altitude) {
				desiredAltitude = altitude;
			}
		}
	\end{lstlisting}
\end{frame}

%%%%%%%%%%%%%%%%%%%%%%%%%%%%%%%%%%%%%%%%%%%%%%%%%%%%

\subsection{Konstruktoren}
\begin{frame}[fragile]
	\frametitle{Konstruktoren}
	
	Autopilot.java:
	\begin{lstlisting}[language=java]
		public class Autopilot {
			private final String planeType;
			
			public Autopilot(String planeType) {
				this.planeType = planeType;
				//andere Initialisierung von anderen Variabeln
			}
		}
	\end{lstlisting}
	
	
	main.java:
	\begin{lstlisting}[language=java]
		Autopilot myAP = new Autopilot("Boing 747");
	\end{lstlisting}
\end{frame}

\section{Zusammenfassung}
\subsection{Was wir heute gemacht haben}
\begin{frame}
	\frametitle{Zusammenfassung}
	\begin{itemize}
		\item Primitive Datentypen
		\item Unterschied zu Referenz-Typen
		\item Operatoren
		\item Attribute, Konstanten, statische Variabeln
		\item Methoden / Konstruktoren
	\end{itemize}
\end{frame}

%Noch fragen Folie
\section{Fragen?}
\subsection*{Fragen} %Für das Design...
\begin{frame}	
	\begin{center}
		\huge{Fragen?}
	\end{center}
\end{frame}



%comic
\begin{frame}[full]
\includegraphics[scale=0.55]{bilder/comics/September-25-2011-18-44-59-aa71ce1bd67502c27bc56a6b8d724897.jpeg}
\end{frame}
\end{document}

\end{document}