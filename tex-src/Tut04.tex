%%% Design basiert auf:
%% LaTeX-Beamer template for KIT design
%% by Erik Burger, Christian Hammer
%% title picture by Klaus Krogmann
%% version 2.1
\documentclass[18pt]{beamer}

\usepackage{templates/beamerthemekit}
\usepackage{graphicx} %because it will be included below
\usepackage{listings}
%\usepackage{wasysym}
\usepackage{color}
\usepackage[T1]{fontenc}
%\usepackage{cmap}
\usepackage{textcomp}
\usepackage[utf8]{inputenc}
%\usepackage{pdfpages}
\definecolor{listinggray}{gray}{0.9}
\definecolor{lbcolor}{rgb}{0.9,0.9,0.9}
\lstset{
        language=Java,
        backgroundcolor=\color{lbcolor},
        tabsize=4,
		keepspaces,
		extendedchars=true,
        rulecolor=\color{black},
        basicstyle=\footnotesize,
        aboveskip=0pt,
        upquote=true,
        columns=fixed,
        showstringspaces=false,
        extendedchars=true,
        breaklines=true,
        frame=single,
        showtabs=false,
        showspaces=false,
        showstringspaces=false,
        identifierstyle=\ttfamily,
        keywordstyle=\color[rgb]{0,0,1},
        commentstyle=\color[rgb]{0.133,0.545,0.133},
        stringstyle=\color[rgb]{0.627,0.126,0.941},
}


%% TITLE PICTURE
\titleimage{frontpic}


% For the title page
\title[Proggen WS11/12]{Programmieren WS 2011/2012}
\subtitle{Tutorium Nr. 1 / 11}
\author{Tobias Sturm} %, David Kulicke
\institute{Zertifizierbare Vertrauenswürdige Informatiksysteme}
\date[23.1.12] %TODO aktualisieren

% the presentation starts here
\begin{document}
\selectlanguage{ngerman}


%title page
\begin{frame}
	\titlepage
\end{frame}


%table of contents
\begin{frame}{Heute:}
%	\setcounter{tocdepth}{1}
	\tableofcontents
\end{frame}

\setbeamercovered{invisible}

%%%%%%%%%%%%%%%%%%%%%%%%%%%%%%%%%%%%%%%%%%%%%%%%%%%%%%%%%%%%%%%%%%%%%%%%%
%%%%%%%%%%%%%%%%%%%%%%%%%%%%%%%%%%%%%%%%%%%%%%%%%%%%%%%%%%%%%%%%%%%%%%%%%
%%%%%%%%%%%%%%%%%%%%%%%%%%%%%%%%%%%%%%%%%%%%%%%%%%%%%%%%%%%%%%%%%%%%%%%%%
\section{Übungsblatt 1}
\subsection{Häufig gemachte Fehler}
\begin{frame}
	\frametitle{Häufig gemachte Fehler}
	\textbf{Code Conventions} \pause
		\begin{itemize}
			\item Attribut- und Methodennamen beginnen immer klein
			\item Klassennamen beginnen immer groß
			\item CamelCase verwenden
			\item Whitespaces (alias Leerzeichen / Tabs) beachten
		\end{itemize}
\end{frame}

%%%%%%%%%%%%%%%%%%%%%%%%%%%%%%%%%%%%%%%%%%%%%%%%%%%%%%%%%%%%%%%%%%%%%%%%%

\section{Code-Conventions}
\subsection{Folien vom Übungsleiter}
\begin{frame}{Whitespaces?}
	$\rightarrow$ Checkstyle1.pdf
\end{frame}

%%%%%%%%%%%%%%%%%%%%%%%%%%%%%%%%%%%%%%%%%%%%%%%%%%%%%%%%%%%%%%%%%%%%%%%%%

\begin{frame}[containsverbatim]
	\frametitle{Code Conventions}
		\begin{lstlisting}
public class Teddy {
	private String name;
	private int alter;
	private boolean istAlt;
	
	public Teddy(String name, int alter) {
		if (alter > 15) {
			istAlt = true;
		} else {
			istAlt = false;
		}
	}
}
		\end{lstlisting}

\end{frame}

%%%%%%%%%%%%%%%%%%%%%%%%%%%%%%%%%%%%%%%%%%%%%%%%%%%%%%%%%%%%%%%%%%%%%%%%%

\begin{frame}[containsverbatim]
	\frametitle{Code Conventions}
		\begin{lstlisting}[showtabs=true, showspaces=true]
public class Teddy {
	private String name;
	private int alter;
	private boolean istAlt;

	public Teddy(String name, int alter) {
		if (alter > 15) {
			istAlt = true;
		} else {
			istAlt = false;
		}
	}
}
		\end{lstlisting}

\end{frame}

%%%%%%%%%%%%%%%%%%%%%%%%%%%%%%%%%%%%%%%%%%%%%%%%%%%%%%%%%%%%%%%%%%%%%%%%%

\begin{frame}[containsverbatim]
	\frametitle{so bitte nicht!}
	
		\emph{Profile.java}
		\begin{lstlisting}
public class Profile {
	class Person {
		String name;
		String nachname;
	}
}
		\end{lstlisting}
		
		Was passiert hier?
\end{frame}
%%%%%%%%%%%%%%%%%%%%%%%%%%%%%%%%%%%%%%%%%%%%%%%%%%%%%%%%%%%%%%%%%%%%%%%%%

\begin{frame}[containsverbatim]
	\frametitle{bitte so!}
	
		\emph{Profile.java}
		\begin{lstlisting}
public class Profile {
	Person owner;
}
		\end{lstlisting}
		
		
		\emph{Person.java}
		\begin{lstlisting}
public class Person {
	String name;
	String nachname;
}
		\end{lstlisting}
\end{frame}

%%%%%%%%%%%%%%%%%%%%%%%%%%%%%%%%%%%%%%%%%%%%%%%%%%%%%%%%%%%%%%%%%%%%%%%%%

\begin{frame}[containsverbatim]
	\frametitle{so bitte nicht!}
	

		\emph{Person.java}
		\begin{lstlisting}
public class Person {
	String name;
	String nachname;
	float geburtsdatum; //z.B. 11.11.2011, hh:mm
}
\end{lstlisting}
\end{frame}

%%%%%%%%%%%%%%%%%%%%%%%%%%%%%%%%%%%%%%%%%%%%%%%%%%%%%%%%%%%%%%%%%%%%%%%%%

\begin{frame}[containsverbatim]
	\frametitle{bitte so!}
	

		\emph{Person.java}
		\begin{lstlisting}
public class Person {
	String name;
	String nachname;
	Timestamp geburtsdatum;
}
\end{lstlisting}

		\emph{Timestamp.java}
		\begin{lstlisting}
public class Timestamp {
	int tag;
	int monat;
	int jahr;
	int stunde;
	int minute;
}
\end{lstlisting}
\end{frame}

%%%%%%%%%%%%%%%%%%%%%%%%%%%%%%%%%%%%%%%%%%%%%%%%%%%%%%%%%%%%%%%%%%%%%%%%%
%%%%%%%%%%%%%%%%%%%%%%%%%%%%%%%%%%%%%%%%%%%%%%%%%%%%%%%%%%%%%%%%%%%%%%%%%
%%%%%%%%%%%%%%%%%%%%%%%%%%%%%%%%%%%%%%%%%%%%%%%%%%%%%%%%%%%%%%%%%%%%%%%%%

\section{set-Methoden}
\subsection{geheime Attribute verändern}
\begin{frame}{Setter-Methoden}
	\textbf{Wozu} braucht man das?
	\pause
	
	Problem: Matrikelnummer-Attribut von Student-Objekt nicht mehr ändern.
	
	$\rightarrow$ private Attribut und get-Methode.
	
	
	
	\pause neues Problem: Wir wollen private Attribute doch ändern!
	
	\pause $\rightarrow$ \textbf{wtf?}
	
\end{frame}


%%%%%%%%%%%%%%%%%%%%%%%%%%%%%%%%%%%%%%%%%%%%%%%%%%%%%%%%%%%%%%%%%%%%%%%%%

\begin{frame}[containsverbatim]
	\frametitle{Setter-Methoden}
		\emph{Teddy.java}
		\begin{lstlisting}
public class Teddy {
	public String name;
	public int alter; //in tagen
	//...
	
	public double berechneTeddyAlter() {
		return Math.sqrt(alter); //!!! alter muss >= 0 sein.
	}
}
		\end{lstlisting}

\end{frame}



%%%%%%%%%%%%%%%%%%%%%%%%%%%%%%%%%%%%%%%%%%%%%%%%%%%%%%%%%%%%%%%%%%%%%%%%%

\begin{frame}[containsverbatim]
	\frametitle{Setter-Methoden}
		\emph{Teddy.java}
		\begin{lstlisting}
public class Teddy {
	private String name;
	private int alter; //in tagen
	//...
	
	public double berechneTeddyAlter() {
		return Math.sqrt(alter); //!!! alter muss >= 0 sein.
	}
	
	public void setAlter(int alter) {
		this.alter = alter; //veraendert das private Attribut
	}
	
	public int getAlter() {
		return alter; //gibt das private Attribut zurueck
	}
}
		\end{lstlisting}

\end{frame}

%%%%%%%%%%%%%%%%%%%%%%%%%%%%%%%%%%%%%%%%%%%%%%%%%%%

\begin{frame}[containsverbatim]
	\frametitle{Setter-Methoden}
		\emph{Teddy.java}
		\begin{lstlisting}
	//...
	public void setAlter(int alter) {
		if (alter >= 0) { //pruefe den Parameter, ob gueltig
			this.alter = alter; //veraendert das private Attribut
		}
	}
	//...
		\end{lstlisting} 
		
		\pause $\rightarrow$ jetzt können wir immer das Teddy-Alter berechnen :)

\end{frame}

%%%%%%%%%%%%%%%%%%%%%%%%%%%%%%%%%%%%%%%%%%%%%%%%%%%

\begin{frame}{Was muss ich mir merken?}
	Für jedes Attribut, das ich \textbf{lesen} will, brauche ich eine getter-Methode
	
	
	Für jedes Attribut, das ich \textbf{schreiben} will, brauche ich eine setter-Methode
\end{frame}


%%%%%%%%%%%%%%%%%%%%%%%%%%%%%%%%%%%%%%%%%%%%%%%%%%%

\begin{frame}[containsverbatim]
	\frametitle{Setter-Methoden}
		\emph{0815 - Getter-Methode}
		\begin{lstlisting}
	private <Datentyp> <attribut>; //deklaration des Attributs
	//...
	public <Datentyp> get<Attribut>() {
		return <attribut>;
	}
	//...
		\end{lstlisting} 
		
	\emph{0815 - Setter-Methode}
		\begin{lstlisting}
	//...
	public void set<Attribut>(<Datentyp> <attribut>) {
		this.<attribut> = <attribut>;
	}
	//...
		\end{lstlisting} 
		
		\pause $\rightarrow$ jetzt können wir immer das Teddy-Alter berechnen :)

\end{frame}

%%%%%%%%%%%%%%%%%%%%%%%%%%%%%%%%%%%%%%%%%%%%%%%%%%%
%%%%%%%%%%%%%%%%%%%%%%%%%%%%%%%%%%%%%%%%%%%%%%%%%%%

\subsection{Aufbau einer Klasse}
\begin{frame}
	\frametitle{Noch mehr Conventions ...}
	\textbf{Aufbau einer Klasse im Quelltext}
		\begin{enumerate}
			\item Klassen-Kopf  \pause
			\item Attribute / Konstanten  \pause
			\item Konstruktor(en)  \pause
			\item getter-/setter-Methoden  \pause
			\item andere Methoden  \pause
		\end{enumerate}
\end{frame}

%%%%%%%%%%%%%%%%%%%%%%%%%%%%%%%%%%%%%%%%%%%%%%%%%%%
%%%%%%%%%%%%%%%%%%%%%%%%%%%%%%%%%%%%%%%%%%%%%%%%%%%
%%%%%%%%%%%%%%%%%%%%%%%%%%%%%%%%%%%%%%%%%%%%%%%%%%%

\section{Wdh: Main Methode}
\subsection{Main Methode}
\begin{frame}{Main-Methode}
	\textbf{Wozu Main Methode?} \pause
	
	$\rightarrow$ Wird aufgerufen, wenn das Programm gestartet wird.\pause
\end{frame}

%%%%%%%%%%%%%%%%%%%%%%%%%%%%%%%%%%%%%%%%%%%%%%%%%%%

\begin{frame}{Main-Methode}
	\textbf{Signatur:}

	public static void main(String[] args) \{
	
	\begin{itemize}
		\item public \pause $\rightarrow$ damit Methode nach außen sichtbar ist \pause
		\item static \pause $\rightarrow$ damit kein Objekt erzeugt werden muss \pause
		\item void \pause $\rightarrow$ weil sie nichts zurück gibt \pause
		\item main \pause $\rightarrow$ ist festgelegt \pause
		\item String[] \pause $\rightarrow$ weil sie mehrere String-Parameter von der Konsole bekommt.
	\end{itemize}
	
	$\rightarrow$ Man braucht pro Programm genau \textbf{eine} Main-Methode!
\end{frame}

%%%%%%%%%%%%%%%%%%%%%%%%%%%%%%%%%%%%%%%%%%%%%%%%%%%
%%%%%%%%%%%%%%%%%%%%%%%%%%%%%%%%%%%%%%%%%%%%%%%%%%%

\section{Wdh.: if / while / for}
\subsection{Wiederholung}
\begin{frame}{Wiederholung if}
	$\rightarrow$ Progr WS10 04Kontrollstrukturen.pdf slide 8 - 14
\end{frame}

%%%%%%%%%%%%%%%%%%%%%%%%%%%%%%%%%%%%%%%%%%%%%%%%%%%
%%%%%%%%%%%%%%%%%%%%%%%%%%%%%%%%%%%%%%%%%%%%%%%%%%%

\section{richtig Kommentieren}
\subsection{richtig Kommentieren}
\begin{frame}{richtig Kommentieren}
	$\rightarrow$ Kommentieren.pdf
\end{frame}

%%%%%%%%%%%%%%%%%%%%%%%%%%%%%%%%%%%%%%%%%%%%%%%%%%%
%%%%%%%%%%%%%%%%%%%%%%%%%%%%%%%%%%%%%%%%%%%%%%%%%%%
%%%%%%%%%%%%%%%%%%%%%%%%%%%%%%%%%%%%%%%%%%%%%%%%%%%
 
 \section{Übungsblatt 2}
 \subsection{Tips}
 \begin{frame}{Tips zum aktuellen Übungsblatt}
 
	\textbf{Was muss man wissen?}
	
	\begin{itemize}
		\item Methoden / Konstruktoren schreiben
		\item Attribute schreiben und verstecken
		\item getter-/setter-Methoden schreiben
		\item CodeConventions / Checkstyle
		\item statische Variablen
		\item if-Anweisungen
		\item eine main-Methode schreiben.
	\end{itemize}
 \end{frame}

%%%%%%%%%%%%%%%%%%%%%%%%%%%%%%%%%%%%%%%%%%%%%%%%%%%
%%%%%%%%%%%%%%%%%%%%%%%%%%%%%%%%%%%%%%%%%%%%%%%%%%%
%%%%%%%%%%%%%%%%%%%%%%%%%%%%%%%%%%%%%%%%%%%%%%%%%%%
 
\section{Aufgabenblatt}
\subsection{Aufgabenblatt}
 
\begin{frame}{Aufgabenblatt}
	Jetzt seid ihr dran.
\end{frame}

%Noch fragen Folie
\section{Fragen?}
\subsection*{Fragen} %Für das Design...
\begin{frame}	
	\begin{center}
		\huge{Fragen?}
	\end{center}
\end{frame}



%comic
\begin{frame}[full]
\includegraphics[scale=0.55]{bilder/comics/September-25-2011-18-44-59-aa71ce1bd67502c27bc56a6b8d724897.jpeg}
\end{frame}
\end{document}

\end{document}