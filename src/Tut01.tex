%%% Design basiert auf:
%% LaTeX-Beamer template for KIT design
%% by Erik Burger, Christian Hammer
%% title picture by Klaus Krogmann
%% version 2.1
\documentclass[18pt]{beamer}

\usepackage{templates/beamerthemekit}
\usepackage{graphicx} %because it will be included below
\usepackage{listings}
%\usepackage{wasysym}
\usepackage{color}
\usepackage[T1]{fontenc}
%\usepackage{cmap}
\usepackage{textcomp}
\usepackage[utf8]{inputenc}
%\usepackage{pdfpages}
\definecolor{listinggray}{gray}{0.9}
\definecolor{lbcolor}{rgb}{0.9,0.9,0.9}
\lstset{
        language=Java,
        backgroundcolor=\color{lbcolor},
        tabsize=4,
		keepspaces,
		extendedchars=true,
        rulecolor=\color{black},
        basicstyle=\footnotesize,
        aboveskip=0pt,
        upquote=true,
        columns=fixed,
        showstringspaces=false,
        extendedchars=true,
        breaklines=true,
        frame=single,
        showtabs=false,
        showspaces=false,
        showstringspaces=false,
        identifierstyle=\ttfamily,
        keywordstyle=\color[rgb]{0,0,1},
        commentstyle=\color[rgb]{0.133,0.545,0.133},
        stringstyle=\color[rgb]{0.627,0.126,0.941},
}


%% TITLE PICTURE
\titleimage{frontpic}


% For the title page
\title[Proggen WS11/12]{Programmieren WS 2011/2012}
\subtitle{Tutorium Nr. 1 / 11}
\author{Tobias Sturm} %, David Kulicke
\institute{Zertifizierbare Vertrauenswürdige Informatiksysteme}
\date[23.1.12] %TODO aktualisieren

% the presentation starts here
\begin{document}
\selectlanguage{ngerman}


%title page
\begin{frame}
	\titlepage
\end{frame}


%table of contents
\begin{frame}{Heute:}
%	\setcounter{tocdepth}{1}
	\tableofcontents
\end{frame}

\setbeamercovered{invisible}

%Ganzer kram der hier eigentlich steht, ist in main.tex

\section{wichtige Dinge}

%%%%%%%%
%Slide 0
\begin{frame}
	\frametitle{Organisatorisches}
	Alternatives Tutorium von mir: \emph{Montag, 17:30, SR -118}
	
	bzw. \emph{Montag, 8:00, SR -107}
\end{frame}

%%%%%%%%
%Slide 1
\begin{frame}
	\frametitle{Ansprechpartner Hilfe}
	\framesubtitle{zu wem rennen, wenn es mal brennt}
	
	\textbf{Das Internet}
	\begin{itemize}
		\item \href{http://stackoverflow.com/}{stackoverflow.com}
		\item \href{http://download.oracle.com/javase/7/docs/api/}{Java 7 API}
	\end{itemize}
	
	\pause
	
	\textbf{Ansprechpartner}
	\begin{enumerate}
		\item Kommilitonen
		\item das Praktomats-Forum
		\item Euer Tutor (das bin ich) - \href{mailto:Tobias.Sturm@student.kit.edu}{Tobias.Sturm@student.kit.edu}
		\item Übungsleiter \href{https://zvi.ipd.kit.edu/21_kelbert.php}{(Dipl.-Inf. Florian Kelbert)} - \href{mailto:florian.kelbert@kit.edu}{florian.kelbert@kit.edu}
	\end{enumerate}
\end{frame}

%%%%%%%%
%Slide 2
\begin{frame}
	\frametitle{Zu erledigen!}
	\framesubtitle{am besten gleich nach dem Tutorium}
	\begin{itemize}
		\item Mir eine Email schreiben
		\item \href{http://vpn.kit.edu}{VPN testen}
		\item Praktomat registrieren
		\item Disclaimer ausfüllen
		\item Vorlesungsseite als Favorit speichern:  \href{https://zvi.ipd.kit.edu/lehre_programmieren_ws11.php}{zvi.ipd.kit.edu/lehre\_programmieren\_ws11.php}
		\item Java+Editor installieren (gleich mehr!)
	\end{itemize}
\end{frame}


%%%%%%%%
%Slide 3

\section{Objekte vs. Klassen}
\begin{frame}
	\frametitle{Objekte vs. Klassen}

	\begin{huge}
		\begin{center}
			Objekte $ \Leftrightarrow $ Klassen
		\end{center}
	\end{huge}
\end{frame}

%%%%%%%%%%%
%slide 4
\begin{frame}
	\frametitle{Objekte vs. Klassen}
	\framesubtitle{was ist ein Objekt?}
	\begin{itemize}
		\pause
		\item Identität \uncover<5->{$ \rightarrow  $ Referenz / Name}
		\pause
		\item Zustand \uncover<6->{$ \rightarrow  $ Attribute}
		\pause
		\item Verhalten \uncover<7->{$ \rightarrow  $ Methoden / Funktionen}
	\end{itemize}
\end{frame}

%%%%%%%%%
%slide 5
\begin{frame}
	\frametitle{Objekte vs. Klassen}
	\framesubtitle{was ist eine Klasse?}
		\pause
		\emph{Bauplan} für gleichartige Objekte. 
		\begin{itemize}
			\item welche Attribute sollen die Objekte haben?
			\item welche Methoden sollen die Objekte haben?
		\end{itemize}
\end{frame}

%%%%%%%%%%%%%%
%slide 6
\section{Java}
\begin{frame}
	\frametitle{Hands On!}
	\framesubtitle{Das erste Java-Programm!}
	
	Demo
	
\end{frame}

%%%%%%%%
%Slide 7
\begin{frame}
	\frametitle{Hands On!}
	\framesubtitle{Das solltet ihr jetzt können bzw. wissen}
	
	\begin{itemize}
		\item Klasse vs. Objekt
		\item Klasse in Java schreiben
	\end{itemize}

	\pause
	\textbf{Das nächste mal}
	\begin{itemize}
		\item Instanzen bzw. Objekte erzeugen
		\item Methode schreiben
		\item Programme ausführen
	\end{itemize}
\end{frame}


%Noch fragen Folie
\section{Fragen?}
\subsection*{Fragen} %Für das Design...
\begin{frame}	
	\begin{center}
		\huge{Fragen?}
	\end{center}
\end{frame}



%comic
\begin{frame}[full]
\includegraphics[scale=0.55]{bilder/comics/September-25-2011-18-44-59-aa71ce1bd67502c27bc56a6b8d724897.jpeg}
\end{frame}
\end{document}

\end{document}