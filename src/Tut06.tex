%%% Design basiert auf:
%% LaTeX-Beamer template for KIT design
%% by Erik Burger, Christian Hammer
%% title picture by Klaus Krogmann
%% version 2.1
\documentclass[18pt]{beamer}

\usepackage{templates/beamerthemekit}
\usepackage{graphicx} %because it will be included below
\usepackage{listings}
%\usepackage{wasysym}
\usepackage{color}
\usepackage[T1]{fontenc}
%\usepackage{cmap}
\usepackage{textcomp}
\usepackage[utf8]{inputenc}
%\usepackage{pdfpages}
\definecolor{listinggray}{gray}{0.9}
\definecolor{lbcolor}{rgb}{0.9,0.9,0.9}
\lstset{
        language=Java,
        backgroundcolor=\color{lbcolor},
        tabsize=4,
		keepspaces,
		extendedchars=true,
        rulecolor=\color{black},
        basicstyle=\footnotesize,
        aboveskip=0pt,
        upquote=true,
        columns=fixed,
        showstringspaces=false,
        extendedchars=true,
        breaklines=true,
        frame=single,
        showtabs=false,
        showspaces=false,
        showstringspaces=false,
        identifierstyle=\ttfamily,
        keywordstyle=\color[rgb]{0,0,1},
        commentstyle=\color[rgb]{0.133,0.545,0.133},
        stringstyle=\color[rgb]{0.627,0.126,0.941},
}


%% TITLE PICTURE
\titleimage{frontpic}


% For the title page
\title[Proggen WS11/12]{Programmieren WS 2011/2012}
\subtitle{Tutorium Nr. 1 / 11}
\author{Tobias Sturm} %, David Kulicke
\institute{Zertifizierbare Vertrauenswürdige Informatiksysteme}
\date[23.1.12] %TODO aktualisieren

% the presentation starts here
\begin{document}
\selectlanguage{ngerman}


%title page
\begin{frame}
	\titlepage
\end{frame}


%table of contents
\begin{frame}{Heute:}
%	\setcounter{tocdepth}{1}
	\tableofcontents
\end{frame}

\setbeamercovered{invisible}

%%%%%%%%%%%%%%%%%%%%%%%%%%%%%%%%%%%%%%%%%%%%%%%%%%%%%%%%%%%%%%%%%%%%%%%%%
%%%%%%%%%%%%%%%%%%%%%%%%%%%%%%%%%%%%%%%%%%%%%%%%%%%%%%%%%%%%%%%%%%%%%%%%%
%%%%%%%%%%%%%%%%%%%%%%%%%%%%%%%%%%%%%%%%%%%%%%%%%%%%%%%%%%%%%%%%%%%%%%%%%
\section{Übungsblatt 2}
\subsection{Häufig gemachte Fehler}
\begin{frame}{ÜB2 - Häufig gemachte Fehler}

		\textbf{Lesbarkeit}
		
		\begin{itemize}
			\item Einrückungen
			\item CamelCase
			\item JCC (Reihnfolge von Attribute, Konstruktoren, Methoden; Klammerungen)
		\end{itemize}
		
		\textbf{Aufgabenstellung}
		
		\begin{itemize}
			\item Methoden-Signaturen
		\end{itemize}
\end{frame}

%%%%%%%%%%%%%%%%%%%%%%%%%%%%%%%%%%%%%%%%%%%%%%%%%%%%%%%%%%%%%%%%%%%%%%%%%

\begin{frame}[containsverbatim]
	\frametitle{Ueb2 - Häufig gemachte Fehler}
		\textbf{Javadoc}
		
		Gar kein Javadoc
		
		\begin{lstlisting}
public char extractChar(int p1, String p2) {
	//...
}		\end{lstlisting}
\end{frame}

%%%%%%%%%%%%%%%%%%%%%%%%%%%%%%%%%%%%%%%%%%%%%%%%%%%%%%%%%%%%%%%%%%%%%%%%%

\begin{frame}[containsverbatim]
	\frametitle{Ueb2 - Häufig gemachte Fehler}
		\textbf{Javadoc}
		
		Kein richtiges Javadoc
		
		\begin{lstlisting}
/** Extrahiert eine Zeichen aus einem String*/
public char extractChar(int p1, String p2) {
	//...
}
		\end{lstlisting}
\end{frame}

%%%%%%%%%%%%%%%%%%%%%%%%%%%%%%%%%%%%%%%%%%%%%%%%%%%%%%%%%%%%%%%%%%%%%%%%%


\begin{frame}[containsverbatim]
	\frametitle{Ueb2 - Häufig gemachte Fehler}
		\textbf{Javadoc}
		
		Unvollständiges Javadoc :-(
		
		\begin{lstlisting}
/**
 * Gibt ein Zeichen eines Strings zurueck
 *
 * @param p1 
 * @param p2 
 * @return
 */
public char extractChar(int p1, String p2) {
	//...
}
		\end{lstlisting}
\end{frame}

%%%%%%%%%%%%%%%%%%%%%%%%%%%%%%%%%%%%%%%%%%%%%%%%%%%%%%%%%%%%%%%%%%%%%%%%%

\begin{frame}[containsverbatim]
\frametitle{Ueb2 - Häufig gemachte Fehler}
		\textbf{Javadoc}
		
		\begin{lstlisting}
/**
 * Gibt ein Zeichen eines Strings zurueck
 *
 * Die Positionsangabe ist 0-basierend.
 * Ist die Positionsangabe ungueltig (<0 oder >= laenge des Strings), wird ein Fehler auf der Konsole ausgegeben.
 *
 * @param p1 Position des Zeichens, das extrahiert werden soll. 0-basiert!
 * @param p2 String aus dem das Zeichen extrahiert werden soll
 * @return Das Zeichen aus dem String an der Stelle p1
 */
public char extractChar(int p1, String p2) {
	//...
}
		\end{lstlisting}
\end{frame}

%%%%%%%%%%%%%%%%%%%%%%%%%%%%%%%%%%%%%%%%%%%%%%%%%%%%%%%%%%%%%%%%%%%%%%%%%
\begin{frame}[containsverbatim]
\frametitle{Ueb2 - Häufig gemachte Fehler}
		\textbf{Design-Entscheidungen}
		
		\begin{lstlisting}
public class Message {
//...	
	public Person getIssuer() {
		return issuer;
	}
	
	public void setIssuer(Person issuer) {
		this.issuer = issuer;
	}
	
	public String getText() {
		return text;
	}
	
	public void setText(String text) {
		this.text = text;
	}
}
		\end{lstlisting}
\end{frame}

%%%%%%%%%%%%%%%%%%%%%%%%%%%%%%%%%%%%%%%%%%%%%%%%%%%%%%%%%%%%%%%%%%%%%%%%%
\begin{frame}[containsverbatim]
\frametitle{Ueb2 - Häufig gemachte Fehler}
		\textbf{Design-Entscheidungen}
		
		\begin{lstlisting}
public class KITBook {
	Person person1;
	Person person2;
	Person person3;
	
	Profile profile1;
	Profile profile2;
	Profile profile3;
	//...
}
		\end{lstlisting}
\end{frame}

%%%%%%%%%%%%%%%%%%%%%%%%%%%%%%%%%%%%%%%%%%%%%%%%%%%%%%%%%%%%%%%%%%%%%%%%%

\begin{frame}[containsverbatim]
\frametitle{Ueb2 - Häufig gemachte Fehler}
		\textbf{Design-Entscheidungen}
		
		\begin{lstlisting}
public class Timestamp {
	int jahr;
	int monat;
	int tag;
	int minuten = 0; //damit, wenn nicht gesetzt 0
	int stunden = 0; // -"-
	
	public Timestamp(int minute, int stunde, int tag, int monat, int jahr) {
		this.jahr = jahr;
		this.monat = monat;
		this.tag = tag;
		this.minuten = minuten;
		this.stunden = stunden;
	}
}
		\end{lstlisting}
\end{frame}

%%%%%%%%%%%%%%%%%%%%%%%%%%%%%%%%%%%%%%%%%%%%%%%%%%%%%%%%%%%%%%%%%%%%%%%%%
%%%%%%%%%%%%%%%%%%%%%%%%%%%%%%%%%%%%%%%%%%%%%%%%%%%%%%%%%%%%%%%%%%%%%%%%%
%%%%%%%%%%%%%%%%%%%%%%%%%%%%%%%%%%%%%%%%%%%%%%%%%%%%%%%%%%%%%%%%%%%%%%%%%

\section{Referenzen}
\subsection{mal ohne Programmierkram}
\begin{frame}{Was sind Referenzen?}
	\textbf{Die gelben Seiten}
	\pause
	
	Klempner: 
	\begin{itemize}
		\item Sanitär- und Heizungsinstallteur Wagner - 0721 / 488723
		\item 24h Heizungs-Notdienst - 0762 / 42237
		\item Rohre \& mehr - 1763 / 447231
	\end{itemize}
	\pause
	
	Mit den Referenzen können wir arbeiten:
	\begin{itemize}
		\item Methodenaufruf: "Hallo '24h Heizungs-Notdienst'? Bitte reparieren Sie meine Heizung"\pause  $\rightarrow$ notdienst.repair();\pause 
		\item getMethode: "Hallo '24h Heizungs-Notdienst'? Was kostet eine Arbeitsstunde?"\pause $\rightarrow$ notdienst.getPrice();\pause
	\end{itemize}
\end{frame}

%%%%%%%%%%%%%%%%%%%%%%%%%%%%%%%%%%%%%%%%%%%%%%%%%%%%%%%%%%%%%%%%%%%%%%%%%

\begin{frame}{Was sind Referenzen?}
	\textbf{Die gelben Seiten}
	
	Klempner: 
	\begin{itemize}
		\item Sanitär- und Heizungsinstallteur Wagner - 0721 / 488723
		\item 24h Heizungs-Notdienst - 0762 / 42237
		\item Rohre \& mehr - 1763 / 447231
	\end{itemize}
	
	Was passiert bei "Hallo '24h Heizungs-Notdienst'? Bitte reparieren Sie meine Heizung"\pause
	\begin{enumerate}
		\item Wir rufen '0762 / 42237'
		\item wir bitten, dass die Heizung repariert wird
		\item Der '24h Heizungs-Notdienst' wird die Heizung reparieren
	\end{enumerate}
\end{frame}

%%%%%%%%%%%%%%%%%%%%%%%%%%%%%%%%%%%%%%%%%%%%%%%%%%%%%%%%%%%%%%%%%%%%%%%%%

\begin{frame}{Was sind Referenzen?}
	\textbf{Die gelben Seiten}
	
	Klempner: 
	\begin{itemize}
		\item Sanitär- und Heizungsinstallteur Wagner - 0721 / 488723
		\item 24h Heizungs-Notdienst - 0762 / 42237
		\item Rohre \& mehr - 1763 / 447231
	\end{itemize}
	
	Wir definieren uns einen neuen Namen in den gelben Seiten:
	
	NachtNotDienst = 24h Heizungs-Notdienst \pause
	
	Wie sieht das in den Gelben Seiten dann aus?
\end{frame}

%%%%%%%%%%%%%%%%%%%%%%%%%%%%%%%%%%%%%%%%%%%%%%%%%%%%%%%%%%%%%%%%%%%%%%%%%

\begin{frame}{Was sind Referenzen?}
	\textbf{Die gelben Seiten}
	
	Klempner: 
	\begin{itemize}
		\item Sanitär- und Heizungsinstallteur Wagner - 0721 / 488723
		\item 24h Heizungs-Notdienst - 0762 / 42237
		\item Rohre \& mehr - 1763 / 447231
		\item NachtNotDienst - 0762 / 42237
	\end{itemize} \pause
	
	Was prüfen wir bei 'NachtNotDienst == 24h Heizungs-Notdienst' ?\pause
	
	'0762 / 42237 == 0762 / 42237', also wahr.
\end{frame}

%%%%%%%%%%%%%%%%%%%%%%%%%%%%%%%%%%%%%%%%%%%%%%%%%%%%%%%%%%%%%%%%%%%%%%%%%


\begin{frame}{Was sind Referenzen?}
	\textbf{Die gelben Seiten}
	
	Klempner: 
	\begin{itemize}
		\item Sanitär- und Heizungsinstallteur Wagner - 0721 / 488723
		\item 24h Heizungs-Notdienst - 0762 / 42237
		\item Rohre \& mehr - 1763 / 447231
		\item NachtNotDienst - 0762 / 42237
	\end{itemize}
	
	Was passiert, wenn wir sagen 'new Klempner("Knut der Heizungsmeister")'?
\end{frame}

%%%%%%%%%%%%%%%%%%%%%%%%%%%%%%%%%%%%%%%%%%%%%%%%%%%%%%%%%%%%%%%%%%%%%%%%%

\begin{frame}{Was sind Referenzen?}
	\textbf{Die gelben Seiten}
	
	Klempner: 
	\begin{itemize}
		\item Sanitär- und Heizungsinstallteur Wagner - 0721 / 488723
		\item 24h Heizungs-Notdienst - 0762 / 42237
		\item Rohre \& mehr - 1763 / 447231
		\item NachtNotDienst - 0762 / 42237
		\item Knut der Heizungsmeister - 0742 / 228723
	\end{itemize}
	
	Telefonnummern werden automatisch vergeben.
\end{frame}

%%%%%%%%%%%%%%%%%%%%%%%%%%%%%%%%%%%%%%%%%%%%%%%%%%%%%%%%%%%%%%%%%%%%%%%%%
%%%%%%%%%%%%%%%%%%%%%%%%%%%%%%%%%%%%%%%%%%%%%%%%%%%%%%%%%%%%%%%%%%%%%%%%%

\subsection{Und zurück zu Java}
\begin{frame}{Immer dran deken:}
	\begin{itemize}
		\item '==' prüft bei Referenz-Typen ob zwei Objekte identisch sind.
		\item Um zu prüfen ob zwei Objekte gleich aber nicht identisch sind, verwendet man 'equals'.\pause
		\item Standartmäßig prüft 'equals' auf Identitätsgleicheit.
		\item Man kann aber seine eigene equals-Methode schreiben.
	\end{itemize}
\end{frame}

\begin{frame}[containsverbatim]
\emph{Teddy.java}
\begin{lstlisting}
public class Teddy {
	private String name;
	
	//hier konstruktor
	
	//hier Getter
	
	public boolean equals(Teddy otherInstance) {
		return otherInstance.getName() == name;
	}
}
\end{lstlisting}

\emph{Main.java}
\begin{lstlisting}
//...
Teddy myTeddy = new Teddy("Bruno");
Teddy anotherTeddy = new Teddy("Bruno");
if (myTeddy.equals(anotherTeddy)) {
	System.out.println("Sie sind gleich");
}
//...
\end{lstlisting}
\end{frame}

\section{Aufgabenblatt}
\subsection{Aufgabenblatt}
\begin{frame}[containsverbatim]
\frametitle{Aufgabenblatt}
Betrachten Sie die Methode
\begin{lstlisting}
public static int[] copy(int[] a) {
	int[] b = new int[a.length];
	for (int i = 0; i < a.length; i++) {
		b[i] = a[i];
	}
	return b;
}
\end{lstlisting}
die eine Kopie des Arrays a als Rückgabewert hat.
\end{frame}

\begin{frame}[containsverbatim] 
\frametitle{Aufgabenblatt}
Was ist zu beachten, wenn die Elemente des Arrays a nicht vom Typ int, sondern vom Typ Point sind? Das heißt, wenn die Methode wie folgt aussieht:
\begin{lstlisting}
public static Point[] copy(Point[] a) {
	Point[] b = new Point[a.length];
	for (int i = 0; i < a.length; i++) {
		b[i] = a[i];
	}
	return b;
}
\end{lstlisting}
\end{frame}

\begin{frame}[containsverbatim]
\frametitle{Aufgabenblatt}
Schreiben Sie eine Methode
\begin{lstlisting}
	public static Point[] deepCopy(Point[] a)
\end{lstlisting}
die eine Kopie des Arrays a als Rückgabewert hat und dabei auch alle im Array a enthaltenen
Objekte kopiert.
\end{frame}

\section{Listen}
\subsection{Wozu?}
\begin{frame}{Listen}
	\textbf{Wozu braucht man Listen?}\pause
	
	Wie Array, nur dass die Größe nicht festgelegt werden muss.
\end{frame}

\subsection{Aufbau}
\begin{frame}{Aufbau / Prinzip einer Liste}
	Eine Liste besteht aus vielen Gliedern. Jedes Glied beinhaltet 
	\begin{itemize}
		\item Nutzdaten
		\item Verweis auf das nächste Glied
	\end{itemize}
	
	siehe dazu Tafel.
\end{frame}

\begin{frame}[containsverbatim]
	\frametitle{Aufbau / Prinzip einer Liste}
	
	\emph{Cell.java}
	\begin{lstlisting}
public class Cell {
	public Object value; //aus Platzgruenden keine setter/getter
	public Cell next;
}
	\end{lstlisting}
	
	\textbf{Aufgabe:} Schreibe eine Methode, die das n-te Element aus einer Liste zurückgibt.
	\begin{lstlisting}
public Object getValue(int index)
	\end{lstlisting}
\end{frame}

\begin{frame}[containsverbatim]
		
	\begin{lstlisting}
public class List {
	private Cell head;
	
	public Object getValue(int index) {
		Cell currentCell = head;
		int currentIndex = 0;
		while (currentCell != null && currentIndex < index) {
			currentCell = currentCell.next;
			currentIndex++;
		}

		if (currentIndex < index) {
			return null;
		} else {
			return currentCell.value;
		}
	}
}
	\end{lstlisting}
\end{frame}

\section{Zusammenfassung}
\subsection{Zusammenfassung}
\begin{frame}{Was haben wir heute gemacht?}
	\begin{itemize}
		\item Übungsblatt2
		\item Referenzen
		\item Listen
	\end{itemize}
\end{frame}

%Noch fragen Folie
\section{Fragen?}
\subsection*{Fragen} %Für das Design...
\begin{frame}	
	\begin{center}
		\huge{Fragen?}
	\end{center}
\end{frame}



%comic
\begin{frame}[full]
\includegraphics[scale=0.55]{bilder/comics/September-25-2011-18-44-59-aa71ce1bd67502c27bc56a6b8d724897.jpeg}
\end{frame}
\end{document}

\end{document}