%%% Design basiert auf:
%% LaTeX-Beamer template for KIT design
%% by Erik Burger, Christian Hammer
%% title picture by Klaus Krogmann
%% version 2.1
\documentclass[18pt]{beamer}

\usepackage{templates/beamerthemekit}
\usepackage{graphicx} %because it will be included below
\usepackage{listings}
%\usepackage{wasysym}
\usepackage{color}
\usepackage[T1]{fontenc}
%\usepackage{cmap}
\usepackage{textcomp}
\usepackage[utf8]{inputenc}
%\usepackage{pdfpages}
\definecolor{listinggray}{gray}{0.9}
\definecolor{lbcolor}{rgb}{0.9,0.9,0.9}
\lstset{
        language=Java,
        backgroundcolor=\color{lbcolor},
        tabsize=4,
		keepspaces,
		extendedchars=true,
        rulecolor=\color{black},
        basicstyle=\footnotesize,
        aboveskip=0pt,
        upquote=true,
        columns=fixed,
        showstringspaces=false,
        extendedchars=true,
        breaklines=true,
        frame=single,
        showtabs=false,
        showspaces=false,
        showstringspaces=false,
        identifierstyle=\ttfamily,
        keywordstyle=\color[rgb]{0,0,1},
        commentstyle=\color[rgb]{0.133,0.545,0.133},
        stringstyle=\color[rgb]{0.627,0.126,0.941},
}


%% TITLE PICTURE
\titleimage{frontpic}


% For the title page
\title[Proggen WS11/12]{Programmieren WS 2011/2012}
\subtitle{Tutorium Nr. 1 / 11}
\author{Tobias Sturm} %, David Kulicke
\institute{Zertifizierbare Vertrauenswürdige Informatiksysteme}
\date[23.1.12] %TODO aktualisieren

% the presentation starts here
\begin{document}
\selectlanguage{ngerman}


%title page
\begin{frame}
	\titlepage
\end{frame}


%table of contents
\begin{frame}{Heute:}
%	\setcounter{tocdepth}{1}
	\tableofcontents
\end{frame}

\setbeamercovered{invisible}

%%%%%%%%%%%%%%%%%%%%%%%%%%%%%%%%%%%%%%%%%%%%%%%%%%%%%%%%%%%%%%%%%%%%%%%%%
%%%%%%%%%%%%%%%%%%%%%%%%%%%%%%%%%%%%%%%%%%%%%%%%%%%%%%%%%%%%%%%%%%%%%%%%%
%%%%%%%%%%%%%%%%%%%%%%%%%%%%%%%%%%%%%%%%%%%%%%%%%%%%%%%%%%%%%%%%%%%%%%%%%
\section*{Papierkram}
\subsection*{Papierkram}
\begin{frame}{Papierkram}
	\begin{itemize}
		\item Die erste Abschlussaufgabe beginnt am \textbf{1.2.2012}
		\item Bis zum 31.1.2012, 23:59 \textbf{müsst} ihr euch für die Prüfung (Nr. 199) anmelden!
		\item Das geht, sobald ihr den Übungsschein habt (frühstens ab heute möglich)	
	\end{itemize}
\end{frame}

%%%%%%%%%%%%%%%%%%%%%%%%%%%%%%%%%%%%%%%%%%%%%%%%%%%%%%%%%%%%%%%%%%%%%%%%%
%%%%%%%%%%%%%%%%%%%%%%%%%%%%%%%%%%%%%%%%%%%%%%%%%%%%%%%%%%%%%%%%%%%%%%%%%
%%%%%%%%%%%%%%%%%%%%%%%%%%%%%%%%%%%%%%%%%%%%%%%%%%%%%%%%%%%%%%%%%%%%%%%%%
\section{Übungsblatt 4}
\subsection*{Wozu war es gut?}
\begin{frame}{Übungsblatt 4}
	\begin{itemize}
		\item Klassenvererbung sinnvoll eingesetzt
		\item nicht nur \emph{sinnfreie} Methoden
	\end{itemize}
\end{frame}

%%%%%%%%%%%%%%%%%%%%%%%%%%%%%%%%%%%%%%%%%%%%%%%%%%%%%%%%%%%%%%%%%%%%%%%%%
\subsection*{Fehler}
\begin{frame}[containsverbatim]
	\frametitle{Was ist hier falsch?}
	\emph{CompositeProfileSimilarity}
	\begin{lstlisting}
public CompositeProfileSimilarity(double lambda,
		ProfileSimilarity m1, ProfileSimilarity m2) {
	this.m1 = m1;
	this.m2 = m2;
	this.lambda = lambda;
}
	\end{lstlisting}
\end{frame}

%%%%%%%%%%%%%%%%%%%%%%%%%%%%%%%%%%%%%%%%%%%%%%%%%%%%%%%%%%%%%%%%%%%%%%%%%
%%%%%%%%%%%%%%%%%%%%%%%%%%%%%%%%%%%%%%%%%%%%%%%%%%%%%%%%%%%%%%%%%%%%%%%%%
%%%%%%%%%%%%%%%%%%%%%%%%%%%%%%%%%%%%%%%%%%%%%%%%%%%%%%%%%%%%%%%%%%%%%%%%%
\section{Exceptions}
\subsection*{Exceptions}
\begin{frame}{Was sind Exceptions?}
	Exceptions sind Ausnahmen.\pause
	\begin{itemize}
		\item Ausnahmen sind Fehler die nicht zu erwarten waren.\pause
		\item Fehler sind Zustände eines Programms, sodass das Programm nicht das gewünschte Verhalten zeigt
	\end{itemize}\pause
	
	Exceptions sind auch:
	\begin{itemize}
		\item Objekte, die man \emph{werfen} und \emph{fangen} kann\pause
		\item Spezielle Programm-Konstrukte für Ausnahmebehandlungen
	\end{itemize}\pause
	
	\textbf{Paradox:} Wenn sie nicht zu erwarten waren, warum kann man sie dann erwarten und fangen?
	
\end{frame}

%%%%%%%%%%%%%%%%%%%%%%%%%%%%%%%%%%%%%%%%%%%%%%%%%%%%%%%%%%%%%%%%%%%%%%%%%

\begin{frame}{Exceptions sind Objekte}
	\textbf{Was für Exceptions gibt es?}
	
	Exception (erbt von Throwable)
	\begin{itemize}
			\item RuntimeException
			\begin{itemize}
				\item NullPointerException
				\item IllegalArgumentException
				\item InvalidStateException
			\end{itemize}
			\item IOException
			\begin{itemize}
				\item FileNotFoundException
			\end{itemize}
	\end{itemize}\pause
	
	Alle Exception-Unterklassen findet man in der Doku:\\
	\url{http://docs.oracle.com/javase/7/docs/api/index.html}
\end{frame}

%%%%%%%%%%%%%%%%%%%%%%%%%%%%%%%%%%%%%%%%%%%%%%%%%%%%%%%%%%%%%%%%%%%%%%%%%
%%%%%%%%%%%%%%%%%%%%%%%%%%%%%%%%%%%%%%%%%%%%%%%%%%%%%%%%%%%%%%%%%%%%%%%%%
\subsection*{Exceptions fangen}
\begin{frame}{Fehler fangen}
	\begin{center}
		\textbf{Demo}
	\end{center}
\end{frame}

%%%%%%%%%%%%%%%%%%%%%%%%%%%%%%%%%%%%%%%%%%%%%%%%%%%%%%%%%%%%%%%%%%%%%%%%%

\begin{frame}[containsverbatim]
	\frametitle{Ausnahmen fangen}
	
	\textbf{Eine \emph{Exception} behandeln}
	\begin{lstlisting}
//normal stuff here
try {
	//critical stuff
} catch (RuntimeException ex) { //maybe another Exception type
	//error-handling
}
//normal stuff here
	\end{lstlisting}
\end{frame}

%%%%%%%%%%%%%%%%%%%%%%%%%%%%%%%%%%%%%%%%%%%%%%%%%%%%%%%%%%%%%%%%%%%%%%%%%

\begin{frame}[containsverbatim]
	\frametitle{Ausnahmen fangen}
	
	\textbf{zwei unterschiedliche \emph{Exceptions} behandeln}
	\begin{lstlisting}
//normal stuff here
try {
	//critical stuff
} catch (IOException ex) { //catch ony IO-Exceptions
	//error-handling #1
} catch (OutOfBounceException ex) { //catch ony OutOfBounce-Exceptions
	//error-handling #2
}
//normal stuff here
	\end{lstlisting}
\end{frame}

%%%%%%%%%%%%%%%%%%%%%%%%%%%%%%%%%%%%%%%%%%%%%%%%%%%%%%%%%%%%%%%%%%%%%%%%%

\begin{frame}[containsverbatim]
	\frametitle{Ausnahmen fangen}

	\textbf{Eine \emph{Exception} behandeln (Ninja-Level)}
	\begin{lstlisting}
//normal stuff here
try {
	//critical stuff
} catch (IOException ex) { //catch ony IO-Exceptions
	//error-handling #1
	System.out.println("Error " + ex.getMessage() + ", in " + printStackTrace());
} finally {
	//tidy up
}
//normal stuff here
	\end{lstlisting}
\end{frame}

%%%%%%%%%%%%%%%%%%%%%%%%%%%%%%%%%%%%%%%%%%%%%%%%%%%%%%%%%%%%%%%%%%%%%%%%%

\begin{frame}[containsverbatim]
	\frametitle{NIEMALS}
	
	\begin{lstlisting}
//normal stuff here
try {
	//critical stuff
} catch (Exception ex) {
	//error-handling
}
//normal stuff here
	\end{lstlisting}
\end{frame}

\begin{frame}{NIEMALS}
	\includegraphics[scale=1]{bilder/catch_em_all.png}
\end{frame}

%%%%%%%%%%%%%%%%%%%%%%%%%%%%%%%%%%%%%%%%%%%%%%%%%%%%%%%%%%%%%%%%%%%%%%%%%
%%%%%%%%%%%%%%%%%%%%%%%%%%%%%%%%%%%%%%%%%%%%%%%%%%%%%%%%%%%%%%%%%%%%%%%%%
\subsection*{Ausnahmen werfen}
\begin{frame}{Wann darf ich eine Ausnahmen werfen?}

	Exceptions werfen, wenn.
	\begin{itemize}
		\item Parameter ungültig
		\item Zustand, den man nicht erreichen darf
		\begin{itemize}
			\item etwas was man braucht, ist nicht da
			\item andere Voraussetzung nicht erfüllt
		\end{itemize}
	\end{itemize}
	
\textbf{Nicht bei Dingen, die man Abfragen kann}\pause

In privaten Methoden muss man die Parameter nicht nochmal prüfen, wenn das in den entsprechenden öffentlichen schon gemacht wurde.
\end{frame}

%%%%%%%%%%%%%%%%%%%%%%%%%%%%%%%%%%%%%%%%%%%%%%%%%%%%%%%%%%%%%%%%%%%%%%%%%

\begin{frame}{Wann darf ich eine Ausnahmen werfen?}
	\textbf{Beispiel \emph{KITBook}}
	
	In der \emph{Shell-Klasse} müssen falsche Benutzereingaben (Parameter fehlt usw) mit If/Switch-Case/... behandelt werden\pause
	
	In \emph{KITBook} kann ein Fehler geworfen werden, wenn z.B. eine Person kein Profil hat (\emph{areFriends}) $\rightarrow$ \emph{PersonNotFoundException} (erbt z.B. von \emph{IllegalArgumentException})\pause
	
	Trotzdem: Shell prüft vor Aufrufen auf \emph{KITBook}, ob alles stimmt.
	
	$\rightarrow$ Doppelte Prüfung von Parameter
	
\end{frame}

\begin{frame}{Wann darf ich eine Ausnahmen werfen?}
	\textbf{Generell...}
		
	...stellen Ausnahmen \textbf{KEINE} Steuerung des Kontrollflusses dar!
\end{frame}

\begin{frame}{Welche Ausnahmen darf ich werfen?}
	\begin{itemize}
		\item eigene Error-Unterklassen
		\item eigene RuntimeException-Unterklassen\pause
		\item passende RuntimeException-Unterklassen\pause
		\begin{itemize}
			\item InvalidStateException
			\item UnsupportedOperationException
			\item NullPointerException
		\end{itemize}
	\end{itemize}
\end{frame}

%%%%%%%%%%%%%%%%%%%%%%%%%%%%%%%%%%%%%%%%%%%%%%%%%%%%%%%%%%%%%%%%%%%%%%%%%

\begin{frame}{Fehler werfen}
	\begin{center}
		\textbf{Demo}
	\end{center}
\end{frame}

%%%%%%%%%%%%%%%%%%%%%%%%%%%%%%%%%%%%%%%%%%%%%%%%%%%%%%%%%%%%%%%%%%%%%%%%%

\begin{frame}[containsverbatim]
	\frametitle{Ausnahmen werfen}

\begin{lstlisting}
if(universeWillExplode = true) { //is there an exception?
	throw new IllegalStateException(); //throw a new Exception
	//method won't be continued. Acts like a return
}
	\end{lstlisting}
\end{frame}

%%%%%%%%%%%%%%%%%%%%%%%%%%%%%%%%%%%%%%%%%%%%%%%%%%%%%%%%%%%%%%%%%%%%%%%%%

\begin{frame}[containsverbatim]
	\frametitle{Ausnahmen werfen}

\begin{lstlisting}
if(universeWillExplode = true) { //is there an exception?
	throw new IllegalStateException("The universise is going to explode!"); //throw a new Exception with message
}
	\end{lstlisting}
\end{frame}

%%%%%%%%%%%%%%%%%%%%%%%%%%%%%%%%%%%%%%%%%%%%%%%%%%%%%%%%%%%%%%%%%%%%%%%%%

\begin{frame}[containsverbatim]
	\frametitle{Eigene Exceptions definieren}
	\emph{MyCustomException.java}
	\begin{lstlisting}
/**
 * This Exception indicates ....
 */
public class MyCustomException extends RuntimeException{
	//Constructor with default message
	public MyCustomException() {
		super("An Mycustom-Exception occured!");
	}
}
	\end{lstlisting}
\end{frame}

%%%%%%%%%%%%%%%%%%%%%%%%%%%%%%%%%%%%%%%%%%%%%%%%%%%%%%%%%%%%%%%%%%%%%%%%%

\begin{frame}[containsverbatim]
	\frametitle{Eigene Ausnahmen werfen}

	\emph{UniverseExplodeException.java} 
	\begin{lstlisting}
public class UniverseExplodeException extends RuntimeException{
	//Constructor with default message
	public UniverseExplodeException() {
		super("The universe will explode!");
	}
}
	\end{lstlisting}
	
	\begin{lstlisting}
if(universeWillExplode = true) { //is there an exception?
	throw new UniverseExplodeException();
}
	\end{lstlisting}
\end{frame}

%%%%%%%%%%%%%%%%%%%%%%%%%%%%%%%%%%%%%%%%%%%%%%%%%%%%%%%%%%%%%%%%%%%%%%%%%

\begin{frame}[containsverbatim]
	\frametitle{Eigene Ausnahmen werfen}

	\textbf{Ausnahmen, die nicht von \emph{RuntimeException} erben, müssen angekündigt werden} (Methodensignatur und Javadoc)
	\begin{lstlisting}
/**
 * The foo method ...
 * @throws UniverseExplodeException when the universe is 
 * going to explode
 **/
public void foo() throws UniverseExplodeException
	if(universeWillExplode = true) { //is there an exception?
		throw new UniverseExplodeException();
	}
}
	\end{lstlisting}
\end{frame}

%%%%%%%%%%%%%%%%%%%%%%%%%%%%%%%%%%%%%%%%%%%%%%%%%%%%%%%%%%%%%%%%%%%%%%%%%

\begin{frame}{Ausnahmen}
	\begin{center}
		\textbf{Fragen zu Ausnahmen?}
	\end{center}
\end{frame}
%%%%%%%%%%%%%%%%%%%%%%%%%%%%%%%%%%%%%%%%%%%%%%%%%%%%%%%%%%%%%%%%%%%%%%%%%
%%%%%%%%%%%%%%%%%%%%%%%%%%%%%%%%%%%%%%%%%%%%%%%%%%%%%%%%%%%%%%%%%%%%%%%%%
%%%%%%%%%%%%%%%%%%%%%%%%%%%%%%%%%%%%%%%%%%%%%%%%%%%%%%%%%%%%%%%%%%%%%%%%%
\section{Tiefensuche}
\subsection*{Tiefensuche}
\begin{frame}{Tiefensuche}
	\emph{Was ist Tiefensuche?}
	
	\emph{Beispiel:} Wir haben Ordner. Ordner beinhalten Dateien und andere Ordner. Wir wollen eine Datei bzw. einen Ordner finden.
	
	$\rightarrow$ siehe Tafel
\end{frame}

\begin{frame}{Aufgabe}
	\textbf{Aufgabe:} Schreibe eine Datei-Tiefensuche.
	
	Die Methode \emph{Contains(String)} soll angeben, ob es eine Datei gibt, oder nicht.
	
	\begin{itemize}
		\item IOElement repräsentiert Ordner und Dateien
		\item Dateien haben keine Kinder $\rightarrow$ getChildren() liefert null.
		\item benutze private Hilfsmethode (\emph{contains(String, IOElement[] in)})
	\end{itemize}
	
	Soll-Ausgabe: true, false, true
\end{frame}

\begin{frame}{Aufgabe}
	\textbf{Aufgabe:} Schreibe eine Datei-Tiefensuche.
	
	Die Methode \emph{getPathTo(String)} soll den Pfad zu einer gegebenen Datei angeben (oder \emph{not found :-(}).
	
	\begin{itemize}
		\item IOElement repräsentiert Ordner und Dateien
		\item Dateien haben keine Kinder $\rightarrow$ getChildren() liefert null
		\item benutze private Hilfsmethode (\emph{contains(String, IOElement[] in)})
	\end{itemize}
	
	\textbf{Soll-Ausgabe:} \\
				\textbackslash Folder1\textbackslash FileA\\
				not found :-(\\
				\textbackslash Folder3\textbackslash Folder4\textbackslash Folder5\textbackslash FileH
\end{frame}

\begin{frame}{Aufgabe}
\textbf{Aufgabe:} Schreibe \emph{getPathTo} so um, dass sie eine Ausnahme wirft, wenn die angegebene Datei nicht gefunden wurde.

	\begin{itemize}
		\item Definiere dazu eine eigene Exception-Klasse, die von \emph{Error} erbt
		\item Throws-Schlüsselwort bei der Methode \emph{getPathTo} nicht vergessen
		\item Javadoc!
		\item Fang die Exception nicht und schau, was passiert.
	\end{itemize}
\end{frame}

%%%%%%%%%%%%%%%%%%%%%%%%%%%%%%%%%%%%%%%%%%%%%%%%%%%%%%%%%%%%%%%%%%%%%%%%%
%%%%%%%%%%%%%%%%%%%%%%%%%%%%%%%%%%%%%%%%%%%%%%%%%%%%%%%%%%%%%%%%%%%%%%%%%
%%%%%%%%%%%%%%%%%%%%%%%%%%%%%%%%%%%%%%%%%%%%%%%%%%%%%%%%%%%%%%%%%%%%%%%%%
\section{Übungsblatt 5}
\subsection*{Hinweise}
\begin{frame}{zum aktuellen ÜB5}

	Übernehmt bitte die Musterlösung (auch wenn sie \emph{doof} ist)!
	
	\begin{itemize}
		\item !Abgabe bis 16.1.2012
		\item Verwendet die Musterlösung
		\item Lasst euch von all dem Mathe-Zeug nicht entmutigen!
		\item \emph{adäquates Verhalten öffentlicher Methoden bei ungültigen Parametern}
		\item Wenn man nicht weiter weiß: Signatur hin schreiben und weiter mit nächster Aufgabe
		\item Macht die Zusatzaufgabe (diesmal unabhängig vom KITBook)
	\end{itemize}
\end{frame}

%%%%%%%%%%%%%%%%%%%%%%%%%%%%%%%%%%%%%%%%%%%%%%%%%%%%%%%%%%%%%%%%%%%%%%%%%
%%%%%%%%%%%%%%%%%%%%%%%%%%%%%%%%%%%%%%%%%%%%%%%%%%%%%%%%%%%%%%%%%%%%%%%%%
%%%%%%%%%%%%%%%%%%%%%%%%%%%%%%%%%%%%%%%%%%%%%%%%%%%%%%%%%%%%%%%%%%%%%%%%%
\section*{Zusammenfassung}
\subsection*{Zusammenfassung}
\begin{frame}{Zusammenfassung}
	\textbf{Was haben wir heute gemacht?}
	
	\begin{itemize}
		\item Exceptions werfen und fangen
		\item kein Gonna catch 'em all!
		\item Eigene Exceptions definieren
		\item Tiefensuche implementiert
	\end{itemize}
\end{frame}

%Noch fragen Folie
\section{Fragen?}
\subsection*{Fragen} %Für das Design...
\begin{frame}	
	\begin{center}
		\huge{Fragen?}
	\end{center}
\end{frame}



%comic
\begin{frame}[full]
\includegraphics[scale=0.55]{bilder/comics/September-25-2011-18-44-59-aa71ce1bd67502c27bc56a6b8d724897.jpeg}
\end{frame}
\end{document}

\end{document}